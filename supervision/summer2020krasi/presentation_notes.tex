% Created 2020-07-31 Fri 13:23
% Intended LaTeX compiler: pdflatex
\documentclass[presentation]{beamer}
\usepackage[utf8]{inputenc}
\usepackage[T1]{fontenc}
\usepackage{graphicx}
\usepackage{grffile}
\usepackage{longtable}
\usepackage{wrapfig}
\usepackage{rotating}
\usepackage[normalem]{ulem}
\usepackage{amsmath}
\usepackage{textcomp}
\usepackage{amssymb}
\usepackage{capt-of}
\usepackage{hyperref}
\usetheme{UoB}
\author{Mark Blyth}
\date{\textit{[2020-08-03 Mon]}}
\title{PRESENTATION NOTES, DON'T DISPLAY THESE!}
\hypersetup{
 pdfauthor={Mark Blyth},
 pdftitle={PRESENTATION NOTES, DON'T DISPLAY THESE!},
 pdfkeywords={},
 pdfsubject={},
 pdfcreator={Emacs 26.3 (Org mode 9.1.9)}, 
 pdflang={English}}
\begin{document}

\maketitle

\section{Last time}
\label{sec:org3f06930}
\begin{frame}[plain,label={sec:orgaa9fe4c}]{Last meeting}
Last time: 

\begin{itemize}
\item The challenges of working with spiking signals
\begin{itemize}
\item Lots of high-frequency energy
\item Typically noisy, both from stochastics and from measurement errors
\item Lots of high-frequency components mean a LP filter will remove the signal, as well as noise, so we can't naively clean up the signal
\item Also means using a truncated Fourier discretisation will be infeasible, since it'll have far too many components to effectively discretise
\end{itemize}

\item Surrogate methods to overcome these:
\begin{itemize}
\item Use a regression model in place of the real data
\item Perform desired analysis on this instead
\item `Desired analysis' will be explained more later, in a CBC context
\item A well-chosen surrogate will filter out all the noise, without losing any signal
\end{itemize}
\end{itemize}
\end{frame}

\begin{frame}[plain,label={sec:orga19232b}]{Surrogates point 1}
\begin{itemize}
\item Given \(y_i = f(x_i) + \varepsilon\), estimate \(f(x)\)
\begin{itemize}
\item We assume there's a `true' underlying signal f(x)
\item This true signal is what the neuron is actually doing, eg. what the membrane potential actually is at the patch clamp location
\item We don't have access to f(x); instead, we get a time-series y\(_{\text{i}}\), of noise-corrupted samples
\item These noise-corrupted samples contain both the actual signal at the given sample time, plus some nuissance variable \(\varepsilon\) from errors in measurement
\item We wish to recover f(x) from these samples, as that's the noise-free, true signal that we're interested in
\begin{itemize}
\item Simply LP filtering would only remove the HF components of f(x) and \(\varepsilon\); instead, we wish to separate the two out into noise and signal
\end{itemize}
\item We can use statistical methods to infer f(x) and \(\varepsilon\)
\item This gives us a clean, noise-free surrogate to perform all the analysis on
\item Surrogate: we use f(x) in place of the real data
\end{itemize}
\end{itemize}
\end{frame}

\begin{frame}[plain,label={sec:org4b2e45f}]{Surrogates point 2}
\begin{itemize}
\item Splines and Gaussian processes are good methods for estimation
\begin{itemize}
\item GPR: mathematically elegant, rigorous method
\begin{itemize}
\item Assume normally distributed \(\varepsilon\)
\item Point estimate: y \textasciitilde{} N(f(x), \(\varepsilon\))
\item Whole function estimate: f(x) \textasciitilde{} GP(mu, Sigma)
\item Whole function estimate is actually a Gaussian distribution over functions
\item The whole function estimate is just a generalisation of the point estimate
\item For a sensible prior, we can then use Bayes to estimate the posterior distribution on f(x)
\item Statistically optimal estimator
\item Downside: for finite data, results are only as good as the priors we use; coming up with good priors is hard
\end{itemize}

\item Splines: simple, effective, less elegant
\begin{itemize}
\item Split f(x) into intervals, and asume f(x) is locally polynomial on any given interval
\item Enforce C\(^{\text{2}}\) smoothness over polynomial sections
\item Polynomials then join up at the edges of each interval; these joinings are called knot points
\item Remaining free parameters are chosen to maximise goodness-of-fit
\item No need to define priors, so it's easier to use on data where choice of priors becomes difficult
\item Knot points are difficult to choose; use Bayesian inference to form a posterior distribution over knots
\end{itemize}
\end{itemize}
\end{itemize}
\end{frame}

\section{Novel discretisations}
\label{sec:org34ac465}
\begin{frame}[plain,label={sec:orgd8a879e}]{Developments since last time}
IMAGE. 
\begin{itemize}
\item Surrogates tested and working
\begin{itemize}
\item GPR works on both real and synthetic data, in cases where the data are sufficiently stationary
\item Free knot splines works always, so use it in cases where the data aren't sufficiently stationary
\end{itemize}

\item Image taken from recent abstract
\begin{itemize}
\item Bayesian free-knot splines
\item Works well -- we can extract the underlying signal near perfectly, even given very noisy observations
\item Three changepoints per period: at the start, top, and end of a spike; here, the signal rapidly changes from slow to fast behaviour
\item These changepoints are the hardest bits to model, and therefore the surrogates are least accurate here (the time between spikes shows a slow, gentle change that's easy to model accurately)
\item Zooming in on one of the changepoints, we see that the surrogate recreates the latent signal nearly exactly; even at the most difficult-to-fit part of the signal, we still get excellent results
\end{itemize}
\end{itemize}
\end{frame}

\begin{frame}[plain,label={sec:orgac8f0c5}]{Developments since last time}
\begin{itemize}
\item Novel discretisations
\begin{itemize}
\item Surrogates only give us a noise filter
\item For some CBC implementations, this is sufficient
\item In CBC cases where we have to do Newton iterations, this isn't useful; we instead need a low-dimensional discretisation
\item We can apply the surrogates ideas to creating discretisations
\end{itemize}

\item In-silico CBC
\begin{itemize}
\item Best way to demonstrate that these methods work, are valuable
\end{itemize}
\end{itemize}
\end{frame}

\begin{frame}[plain,label={sec:orgace675d}]{Discretisations}
\begin{itemize}
\item Discretisation takes a function, projects it onto a set of basis functions
\item Coefficients and basis functions are sufficient to represent the signal
\item Lots of possible choices for basis functions
\begin{itemize}
\item C\(^{\text{infinity}}\) signals can be represented exactly with monomial basis functions (taylor expansion)
\item Periodic signals can be represented exactly with trig basis functions (Fourier series)
\item These are bad choices for neuron CBC -- require lots of coefficients to describe the spiking signals
\end{itemize}
\item We've already met splines; turns out we can define a set of basis functions for splines
\begin{itemize}
\item Can therefore express any spline curve in the above form
\item This means we can discretise with splines too!
\item Splines are a good choice: they provide a nice simple, intuitive model, and don't require many basis functions to get a good approximation
\end{itemize}
\end{itemize}
\end{frame}

\begin{frame}[plain,label={sec:orgd6955eb}]{Splines discretisation}
\begin{itemize}
\item Fit a set of basis functions to initial signal \(f_0(x)\)
\begin{itemize}
\item \alert{Choose a set of knots xi, such that the splines basis b\(_{\text{i}}\)(x) that we construct from knots xi is able to fit the initial signal f\(_{\text{0}}\)(x) as accurately as possible, in the least squares sense}
\item This is actually hard to do -- open research problem
\item Elegant approach: find a Bayesian posterior over \(\xi | data\). Downside: is slow and complicated; need to do MCMC to approximate intractable integral
\item Simple approach 1: put a knot at every datapoint then penalise functional of second derivative, to enforce smoothness. Downside: we end up with huge numbers of knots.
\item Simple approach 2: keep adding knots  until we reach satisfactory results; downside: lower quality fit, no guarantee of low-dimensionality
\end{itemize}
\item My approach: choose the number of knots; numerically optimise knot positions; start from random initial knots; avoid local minima by repeating this lots
\begin{itemize}
\item Downside: need to repeat lots to find global minimum
\item Need to choose the number of knots a priori; algo doesn't work it out for us
\item Upside: quick and easy approach to finding a good set of knots; easiest way to get low-dimensional knot set
\end{itemize}
\end{itemize}
\end{frame}
\begin{frame}[plain,label={sec:org36930a4}]{Splines demo}
\begin{itemize}
\item Splines discretisation works well
\item This example uses just 8 knot points
\begin{itemize}
\item Higher than an 8d discretisation, as we need to add exterior knots so that the basis splines have support across the range of the data
\end{itemize}
\item Reconstructs the latent signal near-perfectly
\end{itemize}
\end{frame}

\begin{frame}[plain,label={sec:org66af2fa}]{Splines vs Fourier}
Also shown: Fourier

\begin{itemize}
\item Visually, splines fits better than Fourier
\item Fourier is harder to fit
\item Too few harmonics and the series can't fit the data
\item Too many harmonics and the series starts fitting the noise as well as the data
\item Not really any sweet spot; no point where the series fits the signal, but averages out the noise
\item This is the usage case for surrogates -- when we have noisy data, but still want to use Fourier with it!
\end{itemize}
\end{frame}

\begin{frame}[plain,label={sec:org0d501d7}]{Goodness-of-fit}
This shows the goodness-of-fit of a splines model with given number of knots, and Fourier series with given number of harmonics

\begin{itemize}
\item No noise, Fitzhugh Nagumo
\item Splines error decays more rapidly than Fourier error
\item Effects become even more dramatic for more neuron-like signals
\item Note though this is the goodness-of-fit of a splines, fourier model on a single signal; doesn't determine how well the splines model generalises to discretising unseen signals, ie. only shows how well the spline model fits a signal to which its basis functions were fitted; using the same basis functions on a signal from a different parameter value might get different goodness-of-fit. Fourier won't have this issue since it uses trig basis all the time
\end{itemize}
\end{frame}

\begin{frame}[plain,label={sec:org6e32cd1}]{Method usage cases}
\begin{itemize}
\item Harmonically forced:
\begin{itemize}
\item When we have a harmonically forced system, we can have a harmonically oscillating control action, and treat the control action as the forcing term
\item In this setup, we can efficiently iterate on the Fourier harmonics, to drive the higher-order harmonics of the control action to zero
\item This necessitates a Fourier projection. No need for a novel discretisation, but we could possibly improve the Fourier discretisation by using a surrogate to first filter off the noise
\end{itemize}

\item Non-harmonically forced:
\begin{itemize}
\item If system is unforced, we apply parameter and control action separately, and need the control action to be zero
\item We can use Newton iterations to solve for the noninvasive control action
\item Since we're doing Newton iterations, we need to work with a low dimensional system, otherwise it'll be impractically slow
\item To have a low-dimensional system, we use a novel discretisation, eg. splines
\end{itemize}
\end{itemize}
\end{frame}
\section{CBC approach}
\label{sec:orgd207c9d}
\begin{frame}[plain,label={sec:orgaffdccc}]{\emph{In-silico} CBC}
Current work: implementing an in-silico CBC simulation
\begin{itemize}
\item Best way to test if the surrogates, discretisations work with CBC is to try using them with CBC!
\end{itemize}
\end{frame}

\begin{frame}[plain,label={sec:orgad84812}]{CBC method POINTS 1, 2}
\begin{itemize}
\item Use PD control
\begin{itemize}
\item Easy, model-free control method
\item Gets good results with a method we could easily use in experiments too
\item Fit control parameters with brute force
\item Easy to simulate, minimal effort in controller design
\end{itemize}

\item As per standard numerical continuation, do a change in variables so that time is in [0,1], and treat period as an extra continuation variable
\begin{itemize}
\item Not necessary with Fourier discretisation
\item Splines knots are like finite-differences or collocation mesh points
\item Time rescaling is necessary with mesh-based methods, as changing the period would effectively move the mesh points relative to the signal
\end{itemize}
\end{itemize}
\end{frame}

\begin{frame}[plain,label={sec:orgb719042}]{CBC method POINTS 3, 4}
\begin{itemize}
\item Non-adaptive mesh
\begin{itemize}
\item Fit knots at the start, keep them in the same position throughout
\item Adaptive mesh would mean re-fitting the knots after a prediction-correction step
\item In terms of code, this is minimal extra effort, but would add a slight fitting overhead
\item Only using non-adaptive mesh because I'm interested to see how well it works
\end{itemize}

\item Use Newton iterations to solve for discretised control target = discretised system output
\begin{itemize}
\item Nothing fancy, just simple, slow Newton with finite differences; able to do this in-silico, but would need more rigorous treatment for experiments
\item Sensible to start off with easy root finding, and develop something fancy (Broyden) later
\item Splines method adds exterior knots, and some of the coefficients are always zero, so they can be removed from the discretisation to speed up the finite-differences Jacobian step; I'm currently being lazy and not doing this
\end{itemize}
\end{itemize}
\end{frame}
\section{Code progress}
\label{sec:org9cec4fb}
\begin{frame}[plain,label={sec:org02409ff}]{Discretisors}
\begin{itemize}
\item Instantiate desired discretisor type with its relevant parameter
\begin{itemize}
\item Fourier: n\(_{\text{harmonics}}\)
\item Splines: knot locations
\item Regardless of discretisor type, can then call discretisor discretise, discretisor undiscretise
\end{itemize}

\item Simple, standard interface to discretisation routines
\begin{itemize}
\item Able to swap between Fourier, Splines with zero effort
\item Allows direct comparison between discretisation methods
\item Could easily implement any other discretisation (eg. wavelets) using the same interface
\end{itemize}

\item Lightly tested:
\begin{itemize}
\item The code runs and produces sensible outputs
\item Haven't tested its ability to generalise to new signals
\item Ie. don't know how well basis funcs fitted to f\(_{\text{0}}\)(x) will work for discretising f\(_{\text{1}}\)(x)
\end{itemize}
\end{itemize}
\end{frame}
\begin{frame}[plain,label={sec:org0030b1b}]{Controllers}
\begin{itemize}
\item Can design controllers with a standard interface too
\begin{itemize}
\item Set the controller type, control target, gains, and the control matrices
\item The controller object handles the rest
\end{itemize}

\item Similarly, can design systems with a standard interface
\begin{itemize}
\item Specify a function that gives the ODE RHS; a list of ODE parameters; the controller
\item Can then run the controlled model for any choice of time range, ICs, pars
\item Subsequent runs optionally start with ICs given by final state of last run, much like a real system
\end{itemize}
\end{itemize}

The point of all these standardised interfaces is that it becomes really easy to swap everything out; eg. apply to different models, different control strategies, different discretisors

Can then run a CBC experiment in less than 10 lines of code; easy to apply, reapply, experiment with
\end{frame}

\begin{frame}[plain,label={sec:org30ead4c}]{Control}
IMAGE

FH system with sine target; looks very reasonable

\begin{itemize}
\item Lightly tested: code runs, results look very reasonable
\begin{itemize}
\item Seems like a sensible output
\end{itemize}
\item Can easily write out the RHS of a PD-controlled FH system w/ sine target; can compare this explicit RHS to the code-generated system, to make sure the code isn't doing anything funny
\begin{itemize}
\item Haven't done this yet
\end{itemize}
\end{itemize}
\end{frame}

\begin{frame}[plain,label={sec:org0e0455e}]{Continuation}
\begin{itemize}
\item Runs a psuedo-arclength secant-predictor Newton-corrector CBC
\item Code is written now
\item Requires a `system': this is just a controlled model [from previous code], with its arguments binded
\item `system' interface will be easy to write, but haven't got round to this yet
\end{itemize}
\end{frame}

\begin{frame}[plain,label={sec:org0ad523b}]{Simulation summary}
\begin{itemize}
\item Results handling is just something to take the set of natural periodic orbits, apply some measure (eg. amplitude), then plot them on a bifurcation diagram.
\item Should be working and tested within a week
\end{itemize}
\end{frame}


\section{Next steps}
\label{sec:orgd47e73c}
\begin{frame}[plain,label={sec:org562ebbc}]{Open questions}
\begin{itemize}
\item Will splines discretisation work?
\begin{itemize}
\item If splines can only model the signal to which the knots were fitted, they won't work for CBC
\item My guess is they will work
\end{itemize}

\item Stationary or adaptive mesh?
\begin{itemize}
\item If splines basis aren't good at generalising, can re-fit knots at each step, much like an adaptive mesh, which would hopefully fix problems
\end{itemize}

\item Efficient solving methods
\begin{itemize}
\item Remove zero-coefficients from discretisation
\item Broyden Jacobian update?
\item Newton-Picard iterations? Ludovic's suggestion of Newton-iterating on unstable coefficients, fixed-point iterating on stable coefficients; reduces the size of the Jacobian / finite differences step
\end{itemize}

\item Can we interface the code with Simulink?
\begin{itemize}
\item Ludovic has a simulink model that would be fun to play with; haven't looked at it yet since I've been testing the CBC codes; would be interesting to try to call the finished code from MATLAB, in which case we might be able to interface the two
\end{itemize}
\end{itemize}
\end{frame}

\begin{frame}[plain,label={sec:orga31b052}]{Next steps}
\begin{itemize}
\item Continuation tutorial paper
\begin{itemize}
\item Haven't touched it recently
\item Making slower progress since I'm trying to get the stuff for this done before the paper deadline
\end{itemize}

\item NODYCON abstract for this: submitted, accepted

\item Conference paper for this: will start on that once the CBC simulation is sorted
\end{itemize}
\end{frame}
\end{document}
