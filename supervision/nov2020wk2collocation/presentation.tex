% Created 2020-11-16 Mon 14:49
% Intended LaTeX compiler: pdflatex
\documentclass[presentation]{beamer}
\usepackage[utf8]{inputenc}
\usepackage[T1]{fontenc}
\usepackage{graphicx}
\usepackage{grffile}
\usepackage{longtable}
\usepackage{wrapfig}
\usepackage{rotating}
\usepackage[normalem]{ulem}
\usepackage{amsmath}
\usepackage{textcomp}
\usepackage{amssymb}
\usepackage{capt-of}
\usepackage{hyperref}
\usetheme{UoB}
\author{Mark Blyth}
\date{\textit{[2020-11-16 Mon]}}
\title{Collocation, collocation, collocation}
\hypersetup{
 pdfauthor={Mark Blyth},
 pdftitle={Collocation, collocation, collocation},
 pdfkeywords={},
 pdfsubject={},
 pdfcreator={Emacs 27.1 (Org mode 9.3)}, 
 pdflang={English}}
\begin{document}

\maketitle

\section{Background}
\label{sec:orgc9b820c}
\begin{frame}[label={sec:orgaea0849}]{Week's work}
\begin{itemize}
\item Collocation methods for standard continuation
\end{itemize}
\vfill
\begin{itemize}
\item Annual review
\end{itemize}
\vfill
\begin{itemize}
\item Brainwave: collocation methods for CBC
\end{itemize}
\end{frame}

\section{Annual review}
\label{sec:org36a7aaf}
\begin{frame}[label={sec:org04ca75d}]{Annual review}
Very positive! Main take-aways:
\vfill
\begin{enumerate}
\item It's worth looking into how collocation could be used in CBC
\item Differentiation matrices, spectral, pseudospectral methods might let us find a Jacobian without finite differences
\item Next project should work towards some grand unifying goal
\begin{itemize}
\item Physiologists have lots of good experimental techniques for studying neurons; a CBC project would need to find a physiologically useful question to answer
\item Alternatively, could explore the theoreticals of CBC -- noise-robustness, efficiency, accuracy, higher-codimension continuation, etc., to work answer an experimental nonlinear dynamics question
\item Challenge is to find an overall question to answer that is useful its target community
\end{itemize}
\end{enumerate}
\end{frame}

\section{Collocation}
\label{sec:orgd0cd04a}
\begin{frame}[label={sec:org13628d2}]{Collocation}
\begin{itemize}
\item Deriving collocation equations for generic boundary value problems
\end{itemize}
\vfill
\begin{itemize}
\item Re-read Kuznetsov Elements; realised I'm making things harder than they need to be
\item Wikipedia suggested scalar coefficients, separate BSpline basis functions for each dimension
\item More standard method: vector coefficients, scalar basis funcs shared across dimensions
\end{itemize}
\vfill
\begin{itemize}
\item Re-deriving the collocation equations with scalar basis funcs
\end{itemize}
\vfill
\begin{itemize}
\item Realising CBC could use collocation
\item Deriving CBC collocation equations
\end{itemize}
\end{frame}

\begin{frame}[label={sec:orgf13b50f}]{Discretisations}
CBC discretisation:
\begin{enumerate}
\item Project the input signal onto a set of basis functions
\item Project the output signal onto a set of basis functions
\item Solve for equality between input and output coefficients
\end{enumerate}
\vfill
AUTOesque discretisation:
\begin{enumerate}
\item Construct a boundary-value problem for the periodic orbit
\item Construct a mesh across the domain of the independent variable
\item Construct an approximate BVP solution-form
\item Find the basis func. coeff's s.t. the BVP is solved exactly at the meshpoints
\end{enumerate}
\end{frame}

\begin{frame}[label={sec:org62c11af},plain]{Generalisations}
Collocation doesn't seem applicable to CBC
\begin{itemize}
\item It's a method for solving differential equations
\item We don't have access to any differential equations, only the IO map
\end{itemize}
\vfill
\begin{itemize}
\item Consider the CBC IO map
\begin{itemize}
\item Maps some periodic control-target to some periodic system response
\item We can consider the controlled system as some nonlinear operator on the space of periodic functions
\item We seek a function that solves the operator's fixed-point problem
\end{itemize}
\end{itemize}
\vfill
\begin{itemize}
\item Consider the boundary value problem
\begin{itemize}
\item The ODE represents some (possibly nonlinear) differential operator
\item We seek a function that satisfies the differential operator, and some boundary conditions
\end{itemize}
\end{itemize}
\vfill
\begin{itemize}
\item More abstract'ly: collocation helps solve an operator problem; CBC gives us an operator problem
\end{itemize}
\end{frame}

\begin{frame}[label={sec:org3192617}]{Collocation for CBC}
Choose some points along our control target; find coefficients s.t. those points remain unchanged by the IO map
\vfill
\begin{itemize}
\item Let \(C_T\) be the space of \(T\)-periodic functions
\item Let \(N:C_T\to C_T\) be a nonlinear operator defined by the controlled system
\item We seek \(x(t)\) s.t. \(N\left(x(Tt)\right) = x(Tt)\), \(t\in[0,1]\)
\item Define a collocation mesh \(\xi_1 = 0 < \xi_2 < \dots < \xi_n = 1\)
\item Assume a solution of form \(x(t) = \sum\beta_iB_i(t)\)
\item Mandate solution correctness at collocation points
\begin{itemize}
\item \(N\left(\sum\beta_iB_i(\xi_i)\right) = \sum\beta_iB_i(\xi_i)\), \(i\in\{1,2,\dots,n\}\)
\end{itemize}
\item Solve for \(\beta_i\) satisfying the above
\end{itemize}

\vfill
We're free to choose basis func's \(B_i\); lots of collocation literature, lots of basis func choices
\end{frame}

\begin{frame}[label={sec:org1ef5b5f},plain]{Collocation vs the standard method}
Standard method: basis func. coeff's are the object of interest; input and output functions are a means to an end
\begin{itemize}
\item We seek the basis function coefficients that remain unchanged when passed through the IO map
\item We don't really care what the input and output functions are, because if their coefficients are the same, so are the signals
\item This ceases to solve the undiscretised problem when we have any discretisation error
\end{itemize}
\vfill
Collocation: coefficients are a means to an end; input and output functions are the object of interest
\begin{itemize}
\item We seek coefficients s.t. chosen points remain unchanged by the IO map
\item Coefficients are just the parameters we adjust to find this fixed-point; we don't care about their values
\item The discretisation can be inexact and we can still find a solution!
\end{itemize}
\end{frame}

\begin{frame}[label={sec:org0e502da}]{Why doubt the standard method?}
\begin{itemize}
\item The standard method is valid when the discretisation and un-discretisation operators are each other's inverse
\begin{itemize}
\item Can transform between functions, discretisations, and back, with zero error
\item We assume the solution to a discretised map is representative of a solution to the undiscretised map
\item We can prove \emph{[quite easily]} that this assumption breaks when there's discretisation error
\end{itemize}
\end{itemize}
\vfill
\begin{itemize}
\item Not been an issue so far, as Fourier discretisation can be made exact to working precision on the comparatively simple signals used so far
\end{itemize}
\vfill
\begin{itemize}
\item Bonus: collocation is thoroughly tried-and-tested; accepted method because it's very accurate and computationally efficient
\end{itemize}
\end{frame}

\begin{frame}[label={sec:org90522b3}]{Challenges of collocation}
\begin{itemize}
\item Noise-robustness
\begin{itemize}
\item We're requiring the signals to be exactly equal at the collocation points; this removes all the noise-filtering abilities of the basis functions
\item Surrogates would fix this!
\item Alternatively: collocate statistically -- model noise and find the statistically optimal coefficients
\end{itemize}
\end{itemize}
\vfill
\begin{itemize}
\item Differentiation
\begin{itemize}
\item Perhaps there's alternative methods to finite-differences?
\item Differentiation matrices? Spectral methods? Generalised secant methods?
\end{itemize}
\end{itemize}
\vfill
Interesting aside: take the infinite limit of number of collocation points; we then get the `minimally invasive' control reformulation I've discussed previously
\end{frame}

\section{Next steps}
\label{sec:orgb2bfe44}
\begin{frame}[label={sec:orgcad41f5}]{Next steps}
\begin{itemize}
\item Take a break from \emph{[but don't abandon!]} standard-continuation, BSpline discretisation using the `standard' CBC discretisation method
\end{itemize}
\vfill
\begin{itemize}
\item Try CBC BSpline discretisation using collocation
\item Compare collocation basis functions
\end{itemize}
\vfill
\begin{itemize}
\item Then\ldots{} numerical methods
\begin{itemize}
\item Efficient collocation-system solvers, Jacobian estimation
\end{itemize}
\end{itemize}
\vfill
Target result: demonstrate efficient CBC discretisation using collocation methods
\end{frame}
\end{document}
