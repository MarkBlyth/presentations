% Created 2020-02-17 Mon 08:50
% Intended LaTeX compiler: pdflatex
\documentclass[11pt]{article}
\usepackage[utf8]{inputenc}
\usepackage[T1]{fontenc}
\usepackage{graphicx}
\usepackage{grffile}
\usepackage{longtable}
\usepackage{wrapfig}
\usepackage{rotating}
\usepackage[normalem]{ulem}
\usepackage{amsmath}
\usepackage{textcomp}
\usepackage{amssymb}
\usepackage{capt-of}
\usepackage{hyperref}
\usepackage[margin=1in]{geometry}
\author{mark}
\date{\today}
\title{}
\hypersetup{
 pdfauthor={mark},
 pdftitle={},
 pdfkeywords={},
 pdfsubject={},
 pdfcreator={Emacs 26.3 (Org mode 9.1.9)}, 
 pdflang={English}}
\begin{document}

\tableofcontents

\newpage
\section{Questions for Krassy}
\label{sec:org0759dda}
\begin{itemize}
\item How do you produce bifurcation diagrams on the surface of a sphere, like those in your paper?

\item How do you find the slow subsystem paths? Trial and error?
\begin{itemize}
\item Given a bifurcation diagram, it seems hard to find paths that actually correspond to bursting. Even finding the right start and end bifurcations and linking them won't necessarily produce a burstable trajectory.
\item Why are we looking at bifurcations on the surface of the sphere? Is it just as a convenient dimensionality reduction? Is there any a priori reason why this surface should have contained the burster path, or was that just a lucky bit of guesswork?
\end{itemize}

\item Where are you getting the parameter values for the bifurcation regions?
\begin{itemize}
\item Golubitsky uses b=3, mu2 = 1/3. Why?
\item This paper fixes something like b=0.75. Why? Surely then it isn't a codim4 unfolding?
\item The paper claims that for small b (here, b=0.75), any fixed neighbourhood (here the unit ball) around the origin of (mu1,mu2,v) space contains all the additional bifurcations gained from the DD-BT pt. Why? How and why was b=0.75 chosen? How about the unit ball?
\item I don't understand why we're able to fix b at some small value and not limit the types of bifurcation that we see; why does any fixed neighbourhood of 0 in (mu1,mu2,v)-space contain all the relevant bifurcations for sufficiently small b?
\end{itemize}

\item How do you expect the model to be used?
\begin{itemize}
\item Would I trial and error some parameter vals from a bifurcation diagram diagram, to find the areas of parameter space that give whatever bursting/excitability dynamics I'm interested in studying?
\item How biologically relevant is this model, being a phenomenological one? Will it be able to tell me much about what a living neuron is doing? How much information can we get using it to describe a living cell, if we don't consider the different ionic currents?
\item Is there any nice methods for relating the parameters and parameter values back to what's going on in the live cell? Any real-world interpretation of them?
\end{itemize}
\end{itemize}

Notes:
\begin{itemize}
\item In the cubic Lienard model, mu1 is the slow subsystem variable (z), and v is the HR-b-like bursting pattern variable
\end{itemize}

\newpage
\section{Continuation software}
\label{sec:orga68b1c0}
\subsection{Notes}
\label{sec:orgb41ee43}
\subsubsection{ODE examples}
\label{sec:org2311a42}
\begin{enumerate}
\item XPP
\label{sec:org387e062}
\item COCO
\label{sec:orgdec56b3}
\item MATCONT
\label{sec:orgda7eee0}
\item PyDSTool
\label{sec:org1da023a}
\end{enumerate}

\subsubsection{Pure AUTO capabilities}
\label{sec:orge1b9861}
\begin{enumerate}
\item Algebraics
\label{sec:org47ccdf1}
\begin{itemize}
\item Compute sol'n families for algebraic eq's of form \(f(u,p)=0\), \(f(\cdot,\cdot) \in \mathbb{R}^n\)
\item Find branch points, and continue them in two or three parameters
\item Find Hopf points, continue them in two parameters, detect criticality, find zero-Hopf, BT, Bautins
\item Find folds, continue in 2 parameters, find cusps, zero-Hopfs, BTs
\item Find branch points, folds, period doubling, Neimark-Sackers, continue these in 2 or 3 params and switch branches at branch points and PD bifs for map fixed points
\item Find extrema of ojective functions along solution families; continue extrema in more params
\end{itemize}

\item Flows
\label{sec:org3bb2fac}
Consider an ODE of form \(u'(t) = f\big(u(t), p\big)\), \(f(\cdot, \cdot),~u(\cdot) \in \mathbb{R}^n\).
AUTO can\ldots{}
\begin{itemize}
\item Compute stable / unstable periodic sol'n families, and their Floquet multipliers
\item Find folds, branch points, period doublings, Neimark-Sackers, along PO families; branch switching at PO and PD bifs
\item Continue folds, PD bifs, NS bifs in two parameters, and detect 1:\{1,2,3,4\} resonances
\item Continuation of fixed-period orbits for sufficiently large periods
\item Follow curves of homoclinic orbits, detect and continue codim-2 bifs using HomCont
\item Find extrema of integral objective functions along a periodic solution family; continue extrema in more parameters
\item Compute sol'n curves on the unit interval, subject to nonlinear BCs and integral conditions; discretisation uses an adaptive-mesh orthogonal collocation
\item Determine fold, branch points along sol'n families to the above BVP
\end{itemize}
\item PDEs
\label{sec:org609284d}
Also some stuff for reaction-diffusion equations.
\end{enumerate}

\subsubsection{Things to put in the paper}
\label{sec:orgfc18e6b}
Table of comparison:
\begin{itemize}
\item Bifurcations it can do, curves it can continue, and the types of system they can use
\item When they fail, crash, etc.
\item Numerical methods they have available
\item How much do the parameters need manually fiddling?
\item Do we need to code or not?
\end{itemize}

When writing, aim it at a biology audience.
Continuation is a sequence of problems - start off at equilibria, then move to tracking codim2 bifurcations, increase the dimension etc.
Make this nice and clear: explain why we're starting off finding any sorts of bifurcations we can, then continuing those to find others.
Aim it at someone that doesn't understand continuation (assume they know what bifurcations are, but not continuation methods for finding them).
A brief section on the maths (eg. why we need to continue from a steady state, and how continuation works) would probably be useful.
\subsubsection{Investigating the HR model}
\label{sec:org1f8ea72}
\begin{enumerate}
\item Simplifying assumptions
\label{sec:org4466835}
\begin{itemize}
\item b is a parameter influencing the bursting and spiking behaviour (frequency of spiking, ability or inability to burst)
\item We want to find the start/stop bifurcations when in a spiking regime, so we fix I=2 to force the neuron to spike
\item Freeze the fast subsystem (so, ignore the slow subsystem)
\item We therefore have two bifurcation parameters - slow subsystem state z, and bursting-spiking parameter b
\end{itemize}
\item Investigation strategy
\label{sec:org1a40d38}
\begin{itemize}
\item Simulate the neuron for a few different b,z, to see what happens
\item It spikes
\item If the neuron can spike there must be a limit cycle; if there's a planar limit cycle, there must be an equilibrium within it
\item We're interested in when this limit cycle appears or disappears; let's start by investigating how its central equilibrium bifurcates
\end{itemize}
\begin{enumerate}
\item Equilibrium bifurcation
\label{sec:org6821a09}
(1) Find the equilibrium
\begin{itemize}
\item Simulate the system to get a (x,y) phase portrait, for arbitrary initial conditions, params
\begin{itemize}
\item Wikipedia says b=3 is a sensible value, so let's use that to start with
\item The simulations seem to show I=2 as being a nice (but arbitrarily chosen!) value, so let's use that too
\item (Emphasise that these were chosen just by playing around with simulations)
\end{itemize}
\item This shows a stable limit cycle
\item Choose some point within the limit cycle and integrate backwards
\item This allows us to find the (unstable!) equilibrium in the middle of the limit cycle
\begin{itemize}
\item For I=2, b=3, other params at wikipedia default, this gives an equilibrium at x,y=1,-4
\end{itemize}
\end{itemize}
(2) Do a bifurcation analysis in Z of this equilibrium 
\begin{itemize}
\item We choose to bifurcate in Z since this is the forcing term applied by the slow subsystem that causes bursting
\item Since we have a 1d slow subsystem, we must have a hysteresis-loop burster; hyseteresis-loops typically have a Z-shaped nullcline, so let's guess that's going to be the case and plot a bifurcation diagram in (z,x) space
\item We get two LPs and two Hopf's; the first of these Hopfs occurs at z<-10; this is outside the expected range of z for a typical HR firing, so we'll ignore this one and focus on the other three bifs
\end{itemize}
(3) Continue the bifurcations in (z,b) space
\begin{itemize}
\item Get confused and give up?
\end{itemize}
\end{enumerate}
\end{enumerate}

\subsubsection{Refs}
\label{sec:org59cce46}
[1] \url{http://www.math.pitt.edu/\~bard/xpp/whatis.html}

[2] K. Engelborghs, T.Luzyanina, G. Samaey, DDE-BIFTOOL v. 2.00: a Matlab package for bifurcation analysis of delay differential equations, Technical Report TW-330, Department of Computer Science, K.U.Leuven, Leuven, Belgium, 2001.

[3] \url{https://www.dropbox.com/s/cx2ex5o4n4q42ov/manual\_v8.pdf?dl=0}

[4] \url{https://github.com/robclewley/pydstool}

[5] \url{https://pydstool.github.io/PyDSTool/FrontPage.html}

\subsection{Tools overview}
\label{sec:orgf02af9e}
\subsubsection{ODEs}
\label{sec:org40095ef}
\begin{enumerate}
\item XPP
\label{sec:org67a7acc}
\begin{enumerate}
\item Overview
\label{sec:orga63382b}
\begin{itemize}
\item Language: C
\item Interface: GUI only
\item Usage: ODEs, DDEs, SDEs, BVPs, difference equations, functional equations
\item License: GNU GPL V3
\end{itemize}
\item Notes
\label{sec:org3e639e7}
The 'classic' simulation and continuation software.
Still sees active use in a large range of nonlinear problems.
Bifurcation (continuation) methods provided by AUTO and HomCont; probably possible to use AUTO by itself, but no one does because it would be very difficult (needs FORTRAN coding), and XPP provides a good interface to do it.
Takes plain-text input files, with equations written out in text, as opposed to being defined by user-written functions like in eg. matlab.
From [1], \ldots{}
Over a dozen different solvers, covering stiff systems, integral equations, etc.
Supports Poincare sections, nullcline plotting, flow fields, etc., so it's good for visualisation, as well as bifurcation analysis.
Can produce animations in it (somehow?).
Since it's so popular, there's a wealth of tutorials available for it.
Somewhat outdated GUI, but it does the job perfectly adequately.
No command line interface.
Buggy, sometimes segfaults.
\item Tutorials
\label{sec:org8475dda}
Comprehensive tutorial provided by Ermentrout here: \url{http://www.math.pitt.edu/\~bard/bardware/tut/start.html\#toc}
\end{enumerate}
\item {\bfseries\sffamily TODO} COCO
\label{sec:org42e13e1}
\begin{enumerate}
\item Overview
\label{sec:orgfde71e9}
\item Notes
\label{sec:orge7da8bd}
\item Tutorials
\label{sec:org1ab73e9}
\end{enumerate}
\item {\bfseries\sffamily TODO} MatCont
\label{sec:org806501d}
\begin{enumerate}
\item Overview
\label{sec:org5920a3e}
\begin{itemize}
\item Language: MATLAB
\item Interface: GUI only, but CL\(_{\text{MatCont}}\) exists as a command-line version
\item Usage: """""TODO""""""
\item License: Creative Commons Attribution-NonCommercial-ShareAlike 3.0 unported
\end{itemize}
\item Notes
\label{sec:orge540e4c}
Also: CL\(_{\text{MatCont}}\) (commandline interface), MatContM (MatCont for maps)
\item Tutorials
\label{sec:orgcc53d85}
\end{enumerate}
\item PyDSTool
\label{sec:orgd7a0dee}
See \href{https://pydstool.github.io/PyDSTool/ProjectOverview.html}{the project overview} for lots of nice interesting things to talk about
\begin{enumerate}
\item Overview
\label{sec:org143f13e}
\begin{itemize}
\item Language: Python3, with options for invoking C, Fortran
\item Interface: scripting only
\item Usage: ODEs, DAEs, discrete maps, and hybrid models thereof; some support for DDEs
\item License: BSD 3-clause
\end{itemize}
\item Notes
\label{sec:org1b50bc9}
Julia DS library is just PyDSTool in a julia wrapper.
Provides a full set of tools for development, simulation, and analysis of dynamical system models.
'supports symbolic math, optimisation, phase plane analysis, continuation and bifurcation analysis, data analysis,' etc. (quoted from [5]).
Easy to build into existing code.
Can reuse bits and pieces (eg. continuation, or modelling) for building more complex software.
\item Tutorials
\label{sec:orgb65370b}
Learn-by-example tutorials provided in the examples directory of the code repo [4], and fairly comprehensive documentation available on the website [5].
\end{enumerate}
\end{enumerate}
\subsubsection{Others}
\label{sec:org4c4d449}
\begin{enumerate}
\item DDE Biftool
\label{sec:org24af428}
\begin{enumerate}
\item Overview
\label{sec:org8c8af7a}
\begin{itemize}
\item Language: MATLAB
\item Interface: Scripting
\item Usage: DDEs, sd-DDEs
\item License: BSD 2-clause
\end{itemize}
\item Notes
\label{sec:orgf60bc1b}
DDE bifurcation analysis only.
Described in detail at \url{http://twr.cs.kuleuven.be/research/software/delay/ddebiftool.shtml} .
Full manual available at [2].
Designed for numerical bifurcation analysis of fixed points and periodic orbits, in constant-delay differential equations, and in state-dependent-delay differential equations.
Uses orthogonal collocation (???) to continue steady states, periodic orbits.
Doesn't provide automatic bifurcation detection, but instead tracks eigenvalue evolution, so that the user can determine bifurcation points.
No simulation ability.
\end{enumerate}
\item Knut
\label{sec:orgbf79300}
\begin{enumerate}
\item Overview
\label{sec:org0109710}
\begin{itemize}
\item Language: C++
\item Interface: GUI, CLI
\item Usage: explicitly time-dependent-delay DDEs
\item License: GNU GPL
\end{itemize}
\item Notes
\label{sec:org944412e}
\begin{enumerate}
\item Features:
\label{sec:orgbf27e1a}
[Info taken verbatim from \url{https://rs1909.github.io/knut/}]:
\begin{itemize}
\item Continuation of periodic orbits along a parameter
\item Floquet multiplier calculations
\item Automatic bifurcation detection
\item Continuation of some bifurcations in 2 parameters
\end{itemize}
\item Differences from DDE Biftool:
\label{sec:org0968ddf}
[Info taken from \url{https://rs1909.github.io/knut/}]:
\begin{itemize}
\item C++ makes it faster than MATLAB
\item Standalone software (no need to install matlab as well)
\item GUI-based, with plaintext input, so no need for any programming skills to use it
\item Only software to calculate quasi-periodic tori
\end{itemize}
\end{enumerate}
\item Tutorials
\label{sec:orge83d5e4}
See reference manual [3] for how-to's
\end{enumerate}
\item PDECONT
\label{sec:org6171898}
\begin{enumerate}
\item Overiew
\label{sec:orgdb251a1}
\begin{itemize}
\item Language: C
\item Interface: combination of C and a config file. Matlab interface appears to exist, but no documentation for how to use it
\item Usage: PDE discretisations, large systems of ODEs
\item License: unspecified (open-source, and free for non-commerial use)
\end{itemize}
\item Notes
\label{sec:org0a805b5}
Huge long documentation file exists, but that's just full of code implementations. 
Couldn't find any clear, straightforward tutorials for using it.
Need to code in C and produce a big config file to use the software.
Even then, I can't tell what the code is actually designed to do\ldots{}
\end{enumerate}
\end{enumerate}
\subsection{Tables}
\label{sec:org1e980f4}
\subsubsection{Point labels}
\label{sec:org6567192}

\begin{center}
\begin{tabular}{lll}
Point & Label & Also known as\\
\hline
EP & Equilibrium & \\
LC & Limit cycle & \\
LP & Limit point & Fold bifurcation, saddle node bifurcation\\
H & Hopf & Andronov-Hopf bifurcation\\
LPC & Limit point of cycles & Fold / saddle node bifurcation of periodics\\
NS & Neimark-Sacker & Torus bifurcation\\
PD & Period doubling & Flip bifurcation\\
BP & Branch point & \\
CP & Cusp bifurcation & \\
BT & Bogdanov-Takens & \\
ZH & Zero-Hopf & Fold-Hopf, Saddle-node Hopf, Gavrilov-Guckenheimer\\
HH & Double Hopf & Hopf-Hopf bifurcation\\
GH & Generalised Hopf & Bautin\\
BPC & Branch point of cycles & \\
CPC & Cusp point of cycles & \\
CH & Chenciner & Generalised Neimark-Sacker bifurcation\\
LPNS & Fold-Neimark-Sacker & \\
PDNS & Flip-Neimark-Sacker & \\
LPPD & Fold-flip & \\
NSNS & Double Neimark-Sacker & \\
GPD & Generalised period doubling & \\
\end{tabular}
\end{center}

(Taken from the \href{http://www.scholarpedia.org/article/MATCONT}{MATCONT Scholarpedia} page)

\subsubsection{{\bfseries\sffamily TODO} Types of curve}
\label{sec:org7daba9e}

\begin{center}
\begin{tabular}{llllll}
Curve label & Curve type & MATCONT & CoCo & AUTO & PyDSTool\\
\hline
EP-C & Equilibrium & y &  & y & y\\
LP-C & Limit point / fold & y &  & y & y\\
H-C1 & Hopf (method 1) & y &  & y & y\\
H-C2 & Hopf (method 2) & - &  & - & y\\
LC-C & Limit cycle curve (family of POs) & y &  & y & y\\
 & Limit point of cycles & y &  & ? & ?\\
 & Period doubling & y &  & y & **\\
 & Neimark-Sacker & y &  & y & **\\
 & Homoclinic to saddle & y &  & y & n\\
 & Homoclinic to saddle-node & y &  & y & n\\
* & Branch point & y &  &  & \\
* & Branch point of cycles & y &  &  & \\
* & ConnectionSaddle & y &  &  & \\
* & ConnectionSaddleNode & y &  &  & \\
* & HomotopySaddle & y &  &  & \\
* & HomotopySaddleNode & y &  &  & \\
* & ConnectionHet & y &  &  & \\
* & HomotopyHet & y &  &  & \\
* & Heteroclinic & y &  &  & \\
\end{tabular}
\end{center}

$\backslash$* What do thes mean? Are they actually a bifurcation curve type?
$\backslash$** PyDSTool seems to have methods to compute these for fixed points of maps; does that mean they're a maps-only type of curve? Note that it lacks documentation and tests/examples about these methods, so maybe they're not implemented?
? indicates that there doesn't appear to be a native way of doing this, however it's possible that there's ways to do it (eg. AUTO97 apparently let's us track LPCs, and PyDSTool let's us define custom curves to follow, so one could possibly construct a customised continuation regime to track limit points of cycles)

\subsubsection{{\bfseries\sffamily TODO} Types of point}
\label{sec:orgf48b850}

\begin{center}
\begin{tabular}{lrllll}
Point type & Codim & MATCONT & CoCo & XPP & PyDSTool\\
\hline
LP & 1 & y &  & y & y\\
H & 1 & y &  & y & y\\
LPC & 1 & y &  &  & y\\
NS & 1 & y &  &  & y\\
Torus bif &  &  &  & y & \\
PD & 1 & y &  & y & y\\
BP & 2 & y &  & y & y\\
CP & 2 & y &  &  & y\\
BT & 2 & y &  &  & y\\
ZH & 2 & y &  &  & y\\
HH & 2 & y &  &  & y\\
GH & 2 & y &  &  & y\\
BPC & 2 & y &  &  & n\\
CPC & 2 & y &  &  & n\\
CH & 2 & y &  &  & n\\
LPNS & 2 & y &  &  & n\\
PDNS & 2 & y &  &  & n\\
LPPD & 2 & y &  &  & n\\
NSNS & 2 & y &  &  & n\\
GPD & 2 & y &  &  & n\\
\end{tabular}
\end{center}

$\backslash$* Are branch points just 'there's a bifurcation here but we don't know what type specifically'? In that case, any bifurcation that occurs, but isn't one of the labelled ones, would still be detected as a BP.
Also see the MATCONT 'objects related to homoclinics to equilibria' table, and resonances, for additional points it can detect

\subsubsection{{\bfseries\sffamily TODO} Available numerical methods}
\label{sec:org1bc89a0}

\begin{center}
\begin{tabular}{lllll}
Method & MATCONT & CoCo & XPP & PyDSTool\\
\hline
 &  &  &  & \\
\end{tabular}
\end{center}

\subsubsection{{\bfseries\sffamily TODO} Types of system they can simulate}
\label{sec:org4e5b641}

\begin{center}
\begin{tabular}{lllll}
System & MATCONT & CoCo & XPP & PyDSTool\\
\hline
ODE &  &  & y & y\\
PDE (discretized) &  &  & y & n\\
DDE &  &  & y & limited\\
SDE &  &  & y & limited\\
DAE &  &  & y & y\\
BVP &  &  & y & n\\
Maps &  &  & y & y\\
Hybrid &  &  & basic (apparently) & y\\
Integral &  &  & y & n\\
Difference &  &  & y & y\\
Functional &  &  & y & n\\
\end{tabular}
\end{center}

\textbf{While XPP is capable of simulating all the noted systems, I don't know if that is literally just XPP simulating them, or also that AUTO is able to run continuations with them}

Aren't difference equations the same as maps?

\subsubsection{{\bfseries\sffamily TODO} Degree of manual fiddling / parameter tuning}
\label{sec:org0e0c936}
\subsubsection{To code or not code?}
\label{sec:org23324af}

\begin{center}
\begin{tabular}{llll}
MATCONT & XPP & PyDSTool & CoCo\\
\hline
No coding necessary & No coding necessary & Coding required (matlab) & Coding required (matlab)\\
\end{tabular}
\end{center}

\subsubsection{License}
\label{sec:org563d0fb}

\begin{center}
\begin{tabular}{llll}
MATCONT & XPP & PyDSTool & CoCo\\
\hline
Creative commons, but requires a matlab license & GNU GPL v3 & BSD 3 clause & None specified; matlab license required\\
\end{tabular}
\end{center}

There might be the option of running matcont or CoCo in GNU Octave, meaning no matlab license is required, but this is not a given.

\subsubsection{{\bfseries\sffamily TODO} Crashing and instability / ease of use}
\label{sec:org5faf5ba}
\subsubsection{{\bfseries\sffamily TODO} Other stuff}
\label{sec:org6b2a0b2}

\begin{center}
\begin{tabular}{lllll}
Thing & MATCONT & CoCo & XPP & PyDSTool\\
\hline
Toolboxes & biomechanical, compneuro, systems biology &  &  & \\
Auto C code generation & Yes, for ODE/ DAE / map simulations &  &  & \\
Bounds safety & Yes, can preserve eg. non-negativity &  &  & \\
Index-free system & Yes, making for clear syntax &  &  & \\
Extensible & Yes, can easily build on the code and expand it &  &  & \\
Heirarchical model composition & Yes &  &  & \\
Events detection & Yes &  &  & \\
Symbolic manipulation & Yes &  &  & \\
Memory management utilities & Yes, inc. \LaTeX{} markup export, smbl conversion &  &  & \\
Parameter estimation / fitting & Yes, toolboxes for that &  &  & \\
\end{tabular}
\end{center}

\subsubsection{{\bfseries\sffamily TODO} PyDSTool vs others}
\label{sec:org4a44701}

\begin{center}
\begin{tabular}{lll}
PyDSTool & XPP & MATCONT\\
\hline
Arbitrarily large systems & No heirarchical composition-based modelling & \\
Wider range of DE RHS, but no stochastics & Supports stochastic RHS & \\
SUpports long names & 9 character max. for names & \\
Scriptable & Not scriptable & \\
Can embed simulations in other environments & Can only use as a standalone box & \\
Limited DDE support & Supports general DDEs & \\
Fewer integrators than XPP & Supports more ODE integrators than PyDSTool & \\
No BVP solver & Has a BVP solver & \\
Slower than XPP, as fast as MATCONT & Written in C / fortran. Fast! & Slower than XPP, as fast as PyDSTool\\
Closer integration with the programming env & Hard to interface with other programming & Harder to integrate with other coding\\
 &  & \\
\end{tabular}
\end{center}
\newpage
\end{document}
