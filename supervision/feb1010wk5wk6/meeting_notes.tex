% Created 2020-02-13 Thu 14:06
% Intended LaTeX compiler: pdflatex
\documentclass[11pt]{article}
\usepackage[utf8]{inputenc}
\usepackage[T1]{fontenc}
\usepackage{graphicx}
\usepackage{grffile}
\usepackage{longtable}
\usepackage{wrapfig}
\usepackage{rotating}
\usepackage[normalem]{ulem}
\usepackage{amsmath}
\usepackage{textcomp}
\usepackage{amssymb}
\usepackage{capt-of}
\usepackage{hyperref}
\usepackage[margin=1in]{geometry}
\author{mark}
\date{\today}
\title{}
\hypersetup{
 pdfauthor={mark},
 pdftitle={},
 pdfkeywords={},
 pdfsubject={},
 pdfcreator={Emacs 26.3 (Org mode 9.1.9)}, 
 pdflang={English}}
\begin{document}

\tableofcontents

\newpage
\section{Questions for Krassy}
\label{sec:org2bd77b7}
\begin{itemize}
\item How are you managing to produce bifurcation diagrams on the surface of a sphere like that?

\item How are you finding these paths? Trial and error?
\begin{itemize}
\item Given a bifurcation diagram, it seems hard to find paths that actually correspond to bursting. Even finding the right start and end bifurcations and linking them won't necessarily produce a burstable trajectory.
\item Why are we looking at bifurcations on the surface of the sphere? Is it just as a convenient dimensionality reduction? Is there any a priori reason why this surface should have contained the burster path, or was that just a lucky bit of guesswork?
\end{itemize}

\item Where are you getting the parameter values for the bifurcation regions?
\begin{itemize}
\item Golubitsky uses b=3, mu2 = 1/3. Why?
\item This paper fixes something like b=0.75. Why? Surely then it isn't a codim4 unfolding?
\item The paper claims that for small b (here, b=0.75), any fixed neighbourhood (here the unit ball) around the origin of (mu1,mu2,v) space contains all the additional bifurcations gained from the DD-BT pt. Why? How and why was b=0.75 chosen? How about the unit ball?
\item I don't understand why we're able to fix b at some small value and not limit the types of bifurcation that we see; why does any fixed neighbourhood of 0 in (mu1,mu2,v)-space contain all the relevant bifurcations for sufficiently small b?
\end{itemize}

\item How am I supposed to use the model?
\begin{itemize}
\item Trial and error some parameter vals from a bif diagram, to find the areas of parameter space that give whatever bursting/excitability dynamics I'm after?
\item Is this actually a useful model, being a phenomenological one? Will it be able to tell me much about what a living neuron is doing? How much information can we really get from a living cell if we're completely ignoring all ionic currents?
\item How will the model actually be useful to me? How do I relate the parameters to what's going on in reality?
\end{itemize}
\end{itemize}

Notes:
\begin{itemize}
\item In the cubic Lienard model, mu1 is the slow subsystem variable (z), and v is the HR-b-like bursting pattern variable
\end{itemize}

\newpage
\section{Continuation software}
\label{sec:org3e728e5}
\subsection{Notes}
\label{sec:org3eecbd7}
\subsubsection{ODE examples}
\label{sec:org0dcab1d}
\begin{enumerate}
\item XPP
\label{sec:orgc764553}
\item COCO
\label{sec:org72f9737}
\item MATCONT
\label{sec:org3eb519c}
\item PyDSTool
\label{sec:org569538b}
\end{enumerate}

\subsubsection{Pure AUTO capabilities}
\label{sec:orgad143cb}
\begin{enumerate}
\item Algebraics
\label{sec:org14c3828}
\begin{itemize}
\item Compute sol'n families for algebraic eq's of form \(f(u,p)=0\), \(f(\cdot,\cdot) \in \mathbb{R}^n\)
\item Find branch points, and continue them in two or three parameters
\item Find Hopf points, continue them in two parameters, detect criticality, find zero-Hopf, BT, Bautins
\item Find folds, continue in 2 parameters, find cusps, zero-Hopfs, BTs
\item Find branch points, folds, period doubling, Neimark-Sackers, continue these in 2 or 3 params and switch branches at branch points and PD bifs for map fixed points
\item Find extrema of ojective functions along solution families; continue extrema in more params
\end{itemize}

\item Flows
\label{sec:org4ec64b3}
Consider an ODE of form \(u'(t) = f\big(u(t), p\big)\), \(f(\cdot, \cdot),~u(\cdot) \in \mathbb{R}^n\).
AUTO can\ldots{}
\begin{itemize}
\item Compute stable / unstable periodic sol'n families, and their Floquet multipliers
\item Find folds, branch points, period doublings, Neimark-Sackers, along PO families; branch switching at PO and PD bifs
\item Continue folds, PD bifs, NS bifs in two parameters, and detect 1:\{1,2,3,4\} resonances
\item Continuation of fixed-period orbits for sufficiently large periods
\item Follow curves of homoclinic orbits, detect and continue codim-2 bifs using HomCont
\item Find extrema of integral objective functions along a periodic solution family; continue extrema in more parameters
\item Compute sol'n curves on the unit interval, subject to nonlinear BCs and integral conditions; discretisation uses an adaptive-mesh orthogonal collocation
\item Determine fold, branch points along sol'n families to the above BVP
\end{itemize}
\item PDEs
\label{sec:org82f5394}
Also some stuff for reaction-diffusion equations.
\end{enumerate}

\subsubsection{Things to put in the paper}
\label{sec:orgd38417b}
Table of comparison:
\begin{itemize}
\item Bifurcations it can do, curves it can continue, and the types of system they can use
\item When they fail, crash, etc.
\item Numerical methods they have available
\item How much do the parameters need manually fiddling?
\item Do we need to code or not?
\end{itemize}

When writing, aim it at a biology audience.
Continuation is a sequence of problems - start off at equilibria, then move to tracking codim2 bifurcations, increase the dimension etc.
Make this nice and clear: explain why we're starting off finding any sorts of bifurcations we can, then continuing those to find others.
Aim it at someone that doesn't understand continuation (assume they know what bifurcations are, but not continuation methods for finding them).
A brief section on the maths (eg. why we need to continue from a steady state, and how continuation works) would probably be useful.
\subsubsection{Investigating the HR model}
\label{sec:org5b879b5}
\begin{enumerate}
\item Simplifying assumptions
\label{sec:org6d3e616}
\begin{itemize}
\item b is a parameter influencing the bursting and spiking behaviour (frequency of spiking, ability or inability to burst)
\item We want to find the start/stop bifurcations when in a spiking regime, so we fix I=2 to force the neuron to spike
\item Freeze the fast subsystem (so, ignore the slow subsystem)
\item We therefore have two bifurcation parameters - slow subsystem state z, and bursting-spiking parameter b
\end{itemize}
\item Investigation strategy
\label{sec:org84a7969}
\begin{itemize}
\item Simulate the neuron for a few different b,z, to see what happens
\item It spikes
\item If the neuron can spike there must be a limit cycle; if there's a planar limit cycle, there must be an equilibrium within it
\item We're interested in when this limit cycle appears or disappears; let's start by investigating how its central equilibrium bifurcates
\end{itemize}
\begin{enumerate}
\item Equilibrium bifurcation
\label{sec:org10505d2}
(1) Find the equilibrium
\begin{itemize}
\item Simulate the system to get a (x,y) phase portrait, for arbitrary initial conditions, params
\begin{itemize}
\item Wikipedia says b=3 is a sensible value, so let's use that to start with
\item The simulations seem to show I=2 as being a nice (but arbitrarily chosen!) value, so let's use that too
\item (Emphasise that these were chosen just by playing around with simulations)
\end{itemize}
\item This shows a stable limit cycle
\item Choose some point within the limit cycle and integrate backwards
\item This allows us to find the (unstable!) equilibrium in the middle of the limit cycle
\begin{itemize}
\item For I=2, b=3, other params at wikipedia default, this gives an equilibrium at x,y=1,-4
\end{itemize}
\end{itemize}
(2) Do a bifurcation analysis in Z of this equilibrium 
\begin{itemize}
\item We choose to bifurcate in Z since this is the forcing term applied by the slow subsystem that causes bursting
\item Since we have a 1d slow subsystem, we must have a hysteresis-loop burster; hyseteresis-loops typically have a Z-shaped nullcline, so let's guess that's going to be the case and plot a bifurcation diagram in (z,x) space
\item We get two LPs and two Hopf's; the first of these Hopfs occurs at z<-10; this is outside the expected range of z for a typical HR firing, so we'll ignore this one and focus on the other three bifs
\end{itemize}
(3) Continue the bifurcations in (z,b) space
\begin{itemize}
\item Get confused and give up?
\end{itemize}
\end{enumerate}
\end{enumerate}

\subsubsection{Refs}
\label{sec:orgdd1c6c0}
[1] \url{http://www.math.pitt.edu/\~bard/xpp/whatis.html}

[2] K. Engelborghs, T.Luzyanina, G. Samaey, DDE-BIFTOOL v. 2.00: a Matlab package for bifurcation analysis of delay differential equations, Technical Report TW-330, Department of Computer Science, K.U.Leuven, Leuven, Belgium, 2001.

[3] \url{https://www.dropbox.com/s/cx2ex5o4n4q42ov/manual\_v8.pdf?dl=0}

[4] \url{https://github.com/robclewley/pydstool}

[5] \url{https://pydstool.github.io/PyDSTool/FrontPage.html}

\subsection{Tools overview}
\label{sec:org88322de}
\subsubsection{ODEs}
\label{sec:org45a2ced}
\begin{enumerate}
\item XPP
\label{sec:orgb510c45}
\begin{enumerate}
\item Overview
\label{sec:orge4fb1e0}
\begin{itemize}
\item Language: C
\item Interface: GUI only
\item Usage: ODEs, DDEs, SDEs, BVPs, difference equations, functional equations
\item License: GNU GPL V3
\end{itemize}
\item Notes
\label{sec:orgd749ee8}
The 'classic' simulation and continuation software.
Still sees active use in a large range of nonlinear problems.
Bifurcation (continuation) methods provided by AUTO and HomCont; probably possible to use AUTO by itself, but no one does because it would be very difficult (needs FORTRAN coding), and XPP provides a good interface to do it.
Takes plain-text input files, with equations written out in text, as opposed to being defined by user-written functions like in eg. matlab.
From [1], \ldots{}
Over a dozen different solvers, covering stiff systems, integral equations, etc.
Supports Poincare sections, nullcline plotting, flow fields, etc., so it's good for visualisation, as well as bifurcation analysis.
Can produce animations in it (somehow?).
Since it's so popular, there's a wealth of tutorials available for it.
Somewhat outdated GUI, but it does the job perfectly adequately.
No command line interface.
Buggy, sometimes segfaults.
\item Tutorials
\label{sec:org4b2154b}
Comprehensive tutorial provided by Ermentrout here: \url{http://www.math.pitt.edu/\~bard/bardware/tut/start.html\#toc}
\end{enumerate}
\item {\bfseries\sffamily TODO} COCO
\label{sec:org24a0163}
\begin{enumerate}
\item Overview
\label{sec:org6399c3a}
\item Notes
\label{sec:orgf61c096}
\item Tutorials
\label{sec:org90c2a67}
\end{enumerate}
\item {\bfseries\sffamily TODO} MatCont
\label{sec:org47b7f5c}
\begin{enumerate}
\item Overview
\label{sec:org9be2ffa}
\begin{itemize}
\item Language: MATLAB
\item Interface: GUI only, but CL\(_{\text{MatCont}}\) exists as a command-line version
\item Usage: """""TODO""""""
\item License: Creative Commons Attribution-NonCommercial-ShareAlike 3.0 unported
\end{itemize}
\item Notes
\label{sec:orgd3125d4}
Also: CL\(_{\text{MatCont}}\) (commandline interface), MatContM (MatCont for maps)
\item Tutorials
\label{sec:org8adee9f}
\end{enumerate}
\item PyDSTool
\label{sec:org71022f0}
See \href{https://pydstool.github.io/PyDSTool/ProjectOverview.html}{the project overview} for lots of nice interesting things to talk about
\begin{enumerate}
\item Overview
\label{sec:orge21218e}
\begin{itemize}
\item Language: Python3, with options for invoking C, Fortran
\item Interface: scripting only
\item Usage: ODEs, DAEs, discrete maps, and hybrid models thereof; some support for DDEs
\item License: BSD 3-clause
\end{itemize}
\item Notes
\label{sec:org2135ff6}
Julia DS library is just PyDSTool in a julia wrapper.
Provides a full set of tools for development, simulation, and analysis of dynamical system models.
'supports symbolic math, optimisation, phase plane analysis, continuation and bifurcation analysis, data analysis,' etc. (quoted from [5]).
Easy to build into existing code.
Can reuse bits and pieces (eg. continuation, or modelling) for building more complex software.
\item Tutorials
\label{sec:org042fbcd}
Learn-by-example tutorials provided in the examples directory of the code repo [4], and fairly comprehensive documentation available on the website [5].
\end{enumerate}
\end{enumerate}
\subsubsection{Others}
\label{sec:orgcb91b37}
\begin{enumerate}
\item DDE Biftool
\label{sec:org4d6adc6}
\begin{enumerate}
\item Overview
\label{sec:org4f14471}
\begin{itemize}
\item Language: MATLAB
\item Interface: Scripting
\item Usage: DDEs, sd-DDEs
\item License: BSD 2-clause
\end{itemize}
\item Notes
\label{sec:orgc60201e}
DDE bifurcation analysis only.
Described in detail at \url{http://twr.cs.kuleuven.be/research/software/delay/ddebiftool.shtml} .
Full manual available at [2].
Designed for numerical bifurcation analysis of fixed points and periodic orbits, in constant-delay differential equations, and in state-dependent-delay differential equations.
Uses orthogonal collocation (???) to continue steady states, periodic orbits.
Doesn't provide automatic bifurcation detection, but instead tracks eigenvalue evolution, so that the user can determine bifurcation points.
No simulation ability.
\end{enumerate}
\item Knut
\label{sec:orgdb296eb}
\begin{enumerate}
\item Overview
\label{sec:org9d35d81}
\begin{itemize}
\item Language: C++
\item Interface: GUI, CLI
\item Usage: explicitly time-dependent-delay DDEs
\item License: GNU GPL
\end{itemize}
\item Notes
\label{sec:org09bae9a}
\begin{enumerate}
\item Features:
\label{sec:org4cc8408}
[Info taken verbatim from \url{https://rs1909.github.io/knut/}]:
\begin{itemize}
\item Continuation of periodic orbits along a parameter
\item Floquet multiplier calculations
\item Automatic bifurcation detection
\item Continuation of some bifurcations in 2 parameters
\end{itemize}
\item Differences from DDE Biftool:
\label{sec:org2e22eed}
[Info taken from \url{https://rs1909.github.io/knut/}]:
\begin{itemize}
\item C++ makes it faster than MATLAB
\item Standalone software (no need to install matlab as well)
\item GUI-based, with plaintext input, so no need for any programming skills to use it
\item Only software to calculate quasi-periodic tori
\end{itemize}
\end{enumerate}
\item Tutorials
\label{sec:orgf684669}
See reference manual [3] for how-to's
\end{enumerate}
\item PDECONT
\label{sec:org0fc41e4}
\begin{enumerate}
\item Overiew
\label{sec:org619a8b0}
\begin{itemize}
\item Language: C
\item Interface: combination of C and a config file. Matlab interface appears to exist, but no documentation for how to use it
\item Usage: PDE discretisations, large systems of ODEs
\item License: unspecified (open-source, and free for non-commerial use)
\end{itemize}
\item Notes
\label{sec:orga45ceeb}
Huge long documentation file exists, but that's just full of code implementations. 
Couldn't find any clear, straightforward tutorials for using it.
Need to code in C and produce a big config file to use the software.
Even then, I can't tell what the code is actually designed to do\ldots{}
\end{enumerate}
\end{enumerate}
\subsection{Tables}
\label{sec:orgbfdd939}
\subsubsection{Point labels}
\label{sec:org1ed6b74}

\begin{center}
\begin{tabular}{lll}
Point & Label & Also known as\\
\hline
EP & Equilibrium & \\
LC & Limit cycle & \\
LP & Limit point & Fold bifurcation, saddle node bifurcation\\
H & Hopf & Andronov-Hopf bifurcation\\
LPC & Limit point of cycles & Fold / saddle node bifurcation of periodics\\
NS & Neimark-Sacker & Torus bifurcation\\
PD & Period doubling & Flip bifurcation\\
BP & Branch point & \\
CP & Cusp bifurcation & \\
BT & Bogdanov-Takens & \\
ZH & Zero-Hopf & Fold-Hopf, Saddle-node Hopf, Gavrilov-Guckenheimer\\
HH & Double Hopf & Hopf-Hopf bifurcation\\
GH & Generalised Hopf & Bautin\\
BPC & Branch point of cycles & \\
CPC & Cusp point of cycles & \\
CH & Chenciner & Generalised Neimark-Sacker bifurcation\\
LPNS & Fold-Neimark-Sacker & \\
PDNS & Flip-Neimark-Sacker & \\
LPPD & Fold-flip & \\
NSNS & Double Neimark-Sacker & \\
GPD & Generalised period doubling & \\
\end{tabular}
\end{center}

(Taken from the \href{http://www.scholarpedia.org/article/MATCONT}{MATCONT Scholarpedia} page)

\subsubsection{{\bfseries\sffamily TODO} Types of curve}
\label{sec:org37dfdd4}

\begin{center}
\begin{tabular}{llllll}
Curve label & Curve type & MATCONT & CoCo & AUTO & PyDSTool\\
\hline
EP-C & Equilibrium & y &  & y & y\\
LP-C & Limit point / fold & y &  & y & y\\
H-C1 & Hopf (method 1) & y &  & y & y\\
H-C2 & Hopf (method 2) & - &  & - & y\\
LC-C & Limit cycle curve (family of POs) & y &  & y & y\\
 & Limit point of cycles & y &  & ? & ?\\
 & Period doubling & y &  & y & **\\
 & Neimark-Sacker & y &  & y & **\\
 & Homoclinic to saddle & y &  & y & n\\
 & Homoclinic to saddle-node & y &  & y & n\\
* & Branch point & y &  &  & \\
* & Branch point of cycles & y &  &  & \\
* & ConnectionSaddle & y &  &  & \\
* & ConnectionSaddleNode & y &  &  & \\
* & HomotopySaddle & y &  &  & \\
* & HomotopySaddleNode & y &  &  & \\
* & ConnectionHet & y &  &  & \\
* & HomotopyHet & y &  &  & \\
* & Heteroclinic & y &  &  & \\
\end{tabular}
\end{center}

$\backslash$* What do thes mean? Are they actually a bifurcation curve type?
$\backslash$** PyDSTool seems to have methods to compute these for fixed points of maps; does that mean they're a maps-only type of curve? Note that it lacks documentation and tests/examples about these methods, so maybe they're not implemented?
? indicates that there doesn't appear to be a native way of doing this, however it's possible that there's ways to do it (eg. AUTO97 apparently let's us track LPCs, and PyDSTool let's us define custom curves to follow, so one could possibly construct a customised continuation regime to track limit points of cycles)

\subsubsection{{\bfseries\sffamily TODO} Types of point}
\label{sec:org4ad15b7}

\begin{center}
\begin{tabular}{lrllll}
Point type & Codim & MATCONT & CoCo & XPP & PyDSTool\\
\hline
LP & 1 & y &  & y & y\\
H & 1 & y &  & y & y\\
LPC & 1 & y &  &  & y\\
NS & 1 & y &  &  & y\\
Torus bif &  &  &  & y & \\
PD & 1 & y &  & y & y\\
BP & 2 & y &  & y & y\\
CP & 2 & y &  &  & y\\
BT & 2 & y &  &  & y\\
ZH & 2 & y &  &  & y\\
HH & 2 & y &  &  & y\\
GH & 2 & y &  &  & y\\
BPC & 2 & y &  &  & n\\
CPC & 2 & y &  &  & n\\
CH & 2 & y &  &  & n\\
LPNS & 2 & y &  &  & n\\
PDNS & 2 & y &  &  & n\\
LPPD & 2 & y &  &  & n\\
NSNS & 2 & y &  &  & n\\
GPD & 2 & y &  &  & n\\
\end{tabular}
\end{center}

$\backslash$* Are branch points just 'there's a bifurcation here but we don't know what type specifically'? In that case, any bifurcation that occurs, but isn't one of the labelled ones, would still be detected as a BP.
Also see the MATCONT 'objects related to homoclinics to equilibria' table, and resonances, for additional points it can detect

\subsubsection{{\bfseries\sffamily TODO} Available numerical methods}
\label{sec:org339ab4a}

\begin{center}
\begin{tabular}{lllll}
Method & MATCONT & CoCo & XPP & PyDSTool\\
\hline
 &  &  &  & \\
\end{tabular}
\end{center}

\subsubsection{{\bfseries\sffamily TODO} Types of system they can simulate}
\label{sec:org07f0886}

\begin{center}
\begin{tabular}{lllll}
System & MATCONT & CoCo & XPP & PyDSTool\\
\hline
ODE &  &  & y & y\\
PDE (discretized) &  &  & y & n\\
DDE &  &  & y & limited\\
SDE &  &  & y & limited\\
DAE &  &  & y & y\\
BVP &  &  & y & n\\
Maps &  &  & y & y\\
Hybrid &  &  & basic (apparently) & y\\
Integral &  &  & y & n\\
Difference &  &  & y & y\\
Functional &  &  & y & n\\
\end{tabular}
\end{center}

\textbf{While XPP is capable of simulating all the noted systems, I don't know if that is literally just XPP simulating them, or also that AUTO is able to run continuations with them}

Aren't difference equations the same as maps?

\subsubsection{{\bfseries\sffamily TODO} Degree of manual fiddling / parameter tuning}
\label{sec:orgdf71dcf}
\subsubsection{To code or not code?}
\label{sec:orgc933be9}

\begin{center}
\begin{tabular}{llll}
MATCONT & XPP & PyDSTool & CoCo\\
\hline
No coding necessary & No coding necessary & Coding required (matlab) & Coding required (matlab)\\
\end{tabular}
\end{center}

\subsubsection{License}
\label{sec:orga2a0013}

\begin{center}
\begin{tabular}{llll}
MATCONT & XPP & PyDSTool & CoCo\\
\hline
Creative commons, but requires a matlab license & GNU GPL v3 & BSD 3 clause & None specified; matlab license required\\
\end{tabular}
\end{center}

There might be the option of running matcont or CoCo in GNU Octave, meaning no matlab license is required, but this is not a given.

\subsubsection{{\bfseries\sffamily TODO} Crashing and instability / ease of use}
\label{sec:org1ff4ff9}
\subsubsection{{\bfseries\sffamily TODO} Other stuff}
\label{sec:org9713b0b}

\begin{center}
\begin{tabular}{lllll}
Thing & MATCONT & CoCo & XPP & PyDSTool\\
\hline
Toolboxes & biomechanical, compneuro, systems biology &  &  & \\
Auto C code generation & Yes, for ODE/ DAE / map simulations &  &  & \\
Bounds safety & Yes, can preserve eg. non-negativity &  &  & \\
Index-free system & Yes, making for clear syntax &  &  & \\
Extensible & Yes, can easily build on the code and expand it &  &  & \\
Heirarchical model composition & Yes &  &  & \\
Events detection & Yes &  &  & \\
Symbolic manipulation & Yes &  &  & \\
Memory management utilities & Yes, inc. \LaTeX{} markup export, smbl conversion &  &  & \\
Parameter estimation / fitting & Yes, toolboxes for that &  &  & \\
\end{tabular}
\end{center}

\subsubsection{{\bfseries\sffamily TODO} PyDSTool vs others}
\label{sec:org934bf57}

\begin{center}
\begin{tabular}{lll}
PyDSTool & XPP & MATCONT\\
\hline
Arbitrarily large systems & No heirarchical composition-based modelling & \\
Wider range of DE RHS, but no stochastics & Supports stochastic RHS & \\
SUpports long names & 9 character max. for names & \\
Scriptable & Not scriptable & \\
Can embed simulations in other environments & Can only use as a standalone box & \\
Limited DDE support & Supports general DDEs & \\
Fewer integrators than XPP & Supports more ODE integrators than PyDSTool & \\
No BVP solver & Has a BVP solver & \\
Slower than XPP, as fast as MATCONT & Written in C / fortran. Fast! & Slower than XPP, as fast as PyDSTool\\
Closer integration with the programming env & Hard to interface with other programming & Harder to integrate with other coding\\
 &  & \\
\end{tabular}
\end{center}
\newpage
\section{Burster bibliography and notes}
\label{sec:org55c1532}
\subsection{LITERATURE SUMMARY}
\label{sec:orgaa90283}
The literature spends a lot of time trying to classify bursting neurons into different causes for bursts.
Bursting requires a fast-slow system.
Rinzel (0) introduces the idea of a frozen fast system.
Here, we take the limit as \(\epsilon \to 0\), such that the slow system stops changing.
We treat the slow system state \(y\) as a bifurcation parameter of the fast system.
The fast system will exhibit a variety of bifurcations under y.
(Eg. a pair of saddle-node bifurcations, in the Fitzhugh-Naugmo model.)
The slow variable, when reintroduced, acts as a driving force, which pushes the fast system over these different bifurcations.

Consider the bifurcation set of a bursting system.
Between each bifurcation point, there exists a stable invariant set.
At the bifurcation point, an invariant set either disappears, or loses stability.
The bursting system will trace a path from one invariant set to the next, as the slow subsystem evolves.
This is all explained nicely in (1).

(1) classifies bursters by the bifurcations at either end of the fast subsystem's hysteresis loop.
(3) tries to improve Rinzel's (1) classification, by explaining all bursters as slices through an unfolding of a bifurcation.
(3) considers all of the either-end bifurcations on a 2-parameter bifurcation diagram.
Any given Rinzel bursting type is given by path / periodic motion / cut across this bifurcation diagram, with the either-end bifurcations being those which the cut passes through at the start and end. 
This also allows the prediction of more burster types.
It is noted that the 2d bifurcation diagram is typical of a system near a codim-3 degenerate Bogdanov-Takens bifurcation.

(4) improves (3)'s classification method slightly, to classify bursters by the codimension of the unfolding in which they fist appear, as well as by the bifurcations.
In doing so, it also classifies bursters from the literature as occuring in codim-3 bifurcation unfoldings.
After (4) was written, psuedo-plateau bursters appeared, which can't be explained in terms of codim-3 unfoldings.
Krassy's paper (2) extends the unfolding classification further, by adding psuedo-plateau bursters into the classification system.
This is done by considering the codim-4 unfolding of a doubly degenerate bogdanov takens singularity.
In studying this unfolding, a slow-subsystem path for psuedo-plaueau bursters is uncovered, as well as suggestions for how the systems bifurcate into regular square-wave (plaueau) bursters.
The new (codim-4) unfolding also contains all the codim-3 bursters, and hence, (probably?) every type of known burster so far.

Krassy's paper (2) provides some nice references for the history of explaning bursting.
The general strategy, as mentioned in Krassy's paper, is to find an unfolding containing any relevant fast-subsystem bifurcations, and a path though the parameter space representing the forcing action of the slow variables.

Krassy's paper uses the cubic Lienard system for the (frozen) fast subsystem, as it is one of the partial unfoldings of a doubly-degenerate Bogdanov Takens bifurcation.
Since the paper only considers the frozen fast subsystem, it doesn't pay much attention to the slow system.
A sinusoidal slow subsystem is suggested in the appendices; this forms a slow-wave burster (autonymously oscillating slow subsystem), however no hysteresis-loop slow subsystem is proposed.
To make the model capable of hysteretic bursting, a different slow subsystem must also be defined.
(5) therefore builds further on the work of (0)-(4), by using the same sort of classification scheme as gets developed, but by also adding in a model for a hysteretic slow subsystem.
(The paper also provides a nice review of all the work up to that point, inc. Krassy's paper.)
(IT ONLY SEEMS TO CONSIDER CODIM-3 [SINGLY] DEGENERATE TB SINGULARITIES; IF SO, THE MODEL CAN'T EXHIBIT KRASSY'S PSUEDO-PLATEAU BURSTING.)

\subsection{\href{https://link.springer.com/article/10.1186/2190-8567-1-12}{Dynamics of plateau bursting in pituitary cells (lots of nice refs)}}
\label{sec:org5b5b97a}
\subsubsection{Reference}
\label{sec:orgc3d0e65}
Teka, Wondimu, et al. "The dynamics underlying pseudo-plateau bursting in a pituitary cell model." The Journal of Mathematical Neuroscience 1.1 (2011): 12.

\subsubsection{BibTeX}
\label{sec:orga05f07b}
@article\{teka2011dynamics,
  title=\{The dynamics underlying pseudo-plateau bursting in a pituitary cell model\},
  author=\{Teka, Wondimu and Tabak, Jo\{$\backslash$"e\}l and Vo, Theodore and Wechselberger, Martin and Bertram, Richard\},
  journal=\{The Journal of Mathematical Neuroscience\},
  volume=\{1\},
  number=\{1\},
  pages=\{12\},
  year=\{2011\},
  publisher=\{Springer\}
\}

\subsubsection{Abstract}
\label{sec:org261622f}
Pituitary cells of the anterior pituitary gland secrete hormones in
response to patterns of electrical activity. Several types of
pituitary cells produce short bursts of electrical activity which are
more effective than single spikes in evoking hormone release. These
bursts, called pseudo-plateau bursts, are unlike bursts studied
mathematically in neurons (plateau bursting) and the standard
fast-slow analysis used for plateau bursting is of limited use. Using
an alternative fast-slow analysis, with one fast and two slow
variables, we show that pseudo-plateau bursting is a canard-induced
mixed mode oscillation. Using this technique, it is possible to
determine the region of parameter space where bursting occurs as well
as salient properties of the burst such as the number of spikes in the
burst. The information gained from this one-fast/two-slow
decomposition complements the information obtained from a
two-fast/one-slow decomposition.

\subsubsection{Summary}
\label{sec:orga2d8f1b}
Neurons tend to burst because it's a more effective way of triggering hormone / neurotransmitter release than individual spikes.
This paper looks at different mechanisms to bursting.
Also contains lots of nice useful references about bursting!

\subsection{\href{https://www.sciencedirect.com/science/article/pii/S0166223696100709}{Neurons tend to burst because it's a more effective way of triggering hormone / neurotransmitter release than individual spikes}}
\label{sec:org1b5ff53}
\subsubsection{Reference}
\label{sec:org46dac86}
Lisman, John E. "Bursts as a unit of neural information: making unreliable synapses reliable." Trends in neurosciences 20.1 (1997): 38-43.

\subsubsection{BibTeX}
\label{sec:org875291b}
@article\{lisman1997bursts,
  title=\{Bursts as a unit of neural information: making unreliable synapses reliable\},
  author=\{Lisman, John E\},
  journal=\{Trends in neurosciences\},
  volume=\{20\},
  number=\{1\},
  pages=\{38--43\},
  year=\{1997\},
  publisher=\{Elsevier\}
\}

\subsubsection{Abstract}
\label{sec:org05756cf}
Several lines of evidence indicate that brief ( < 25 ms) bursts of
high-frequency firing have special importance in brain function.
Recent work shows that many central synapses are surprisingly
unreliable at signaling the arrival of single presynaptic action
potentials to the postsynaptic neuron. However, bursts are reliably
signaled because transmitter release is facilitated. Thus, these
synapses can be viewed as filters that transmit bursts, but filter out
single spikes. Bursts appear to have a special role in synaptic
plasticity and information processing. In the hippocampus, a single
burst can produce long-term synaptic modifications. In brain
structures whose computational role is known, action potentials that
arrive in bursts provide more-precise information than action
potentials that arrive singly. These results, and the requirement for
multiple inputs to fire a cell suggest that the best stimulus for
exciting a cell (that is, a neural code) is coincident bursts.

\subsubsection{Summary}
\label{sec:org191b9eb}
Synapses are unreliable, and bursting is the best way to get a signal to cross them.
Acts as a filter and stuff.
Lots of relevant neural information.

\subsection{(0) \href{https://link.springer.com/content/pdf/10.1007/BFb0074739.pdf}{Rinzel's introduction of the fast-slow freezing method to explain bursting}}
\label{sec:org39f18c7}
\subsubsection{Reference}
\label{sec:org9ea909d}
Rinzel, John. "Bursting oscillations in an excitable membrane model." Ordinary and partial differential equations. Springer, Berlin, Heidelberg, 1985. 304-316.

\subsubsection{BibTeX}
\label{sec:orgf55bee3}
@incollection\{rinzel1985bursting,
  title=\{Bursting oscillations in an excitable membrane model\},
  author=\{Rinzel, John\},
  booktitle=\{Ordinary and partial differential equations\},
  pages=\{304--316\},
  year=\{1985\},
  publisher=\{Springer\}
\}

\subsubsection{Abstract}
\label{sec:org8fcdec8}
Various nerve, muscle, and secretory cells exhibit complex electrical
activity which has been observed experimentally by using intracellular
electrodes to monitor the dynamics of the potential across the cell
membrane. Such activity may include single spikes (time scale, msec.)
in response to brief stimuli, repetitive spiking for a maintained
input, and repetitive bursts of spikes (time scale, sec) which may be
endogenous and modulated by chemical (e.g. hormonal) or electrical
stimuli. Pancreatic B­cells respond with periodic bursting in the
presence of glucose (3,13) and this activity is correlated with their
release of insulin (18). Figure 1 illustrates computed solutions of a
theoretical model (4) for such electrical behavior. The mathematical
model (based upon a biophysical model (2)) is an adapted and expanded
version of the classical Hodgkin­Huxley (11) description of nerve
excitability and involves five first­order nonlinear ordinary
differential equations. The time course of membrane potential V(Fig.
1, upper) exhibits spikes of roughly constant size (30­40mV) which
appear to ride on a plateau potential of approximately ­40 mV.
Following each "active phase" of spiking is a "silent phase" where V
slowly increases. The intracellular free calcium concentration Ca
(Fig. 1, lower) slowly increases (on the average) during the active
phase and slowly decreases during the silent phase. The dynamics of Ca
determine the time scale of the bursts. In this paper we present an
analysis and qualitative viewpoint of bursting for the Chay­Keizer
(C­K) theoretical model. We exploit the slow behavior of Ca by first
considering Ca as a parameter and studying its influence on the faster
spike­generating subsystem. Such spike generation dynamics are first
illustrated (Section 2) for a simplified model of excitable membrane
activity with Ca fixed. This two­variable, reduced HH, model yields
single spike and repetitive spike activity such as seen in the active
phase of bursting. In some parameter ranges it exhibits bistability in
which V may rest at a lower stable steady state or oscillate stably
around an upper (unstable) steady state. This latter behavior is also
in the repertoire of the four­variable HH subsystem in the C­K model
and it corresponds to the silent and active phases. Next we append to
the excitation subsystem the slow dynamics of Ca to account for
bursting.

\subsection{(1) \href{http://www-sop.inria.fr/members/Mathieu.Desroches/files/Rinzel\_ICM1986.pdf}{Rinzel classifying bursting mechanisms in terms of the bifurcations exhibited by a neuron (intuitive description of fast-slow burster dynamics)}}
\label{sec:org561c4c5}
\subsubsection{Reference}
\label{sec:org7aa4776}
Rinzel, John. "A formal classification of bursting mechanisms in excitable systems." Mathematical topics in population biology, morphogenesis and neurosciences. Springer, Berlin, Heidelberg, 1987. 267-281.

\subsubsection{BibTeX}
\label{sec:orgfcff1f8}
@incollection\{rinzel1987formal,
  title=\{A formal classification of bursting mechanisms in excitable systems\},
  author=\{Rinzel, John\},
  booktitle=\{Mathematical topics in population biology, morphogenesis and neurosciences\},
  pages=\{267--281\},
  year=\{1987\},
  publisher=\{Springer\}
\}

\subsubsection{Abstract}
\label{sec:org7263abc}
Burst activity is characterized by slowly alternating phases of near
steady state behavior and trains of rapid spike-like oscillations;
examples of bursting patterns are shown in Fig. 2. These two phases
have been called the silent and active phases respectively [2], In the
case of electrical activity of biological membrane systems the slow
time scale of bursting is on the order of tens of seconds while the
spikes have millisecond time scales. In our study of several specific
models for burst activity we have identified a number of different
mechanisms for burst generation (which are characteristic of classes
of models). We will describe qualitatively some of these mechanisms by
way of the schematic diagrams in Fig. 1.

\subsubsection{Summary}
\label{sec:orgde798c7}
One of the original papers on bursting dynamics. 
Explains bursting intuitively, in terms of fast-slow systems.

\subsection{(2) \href{https://research-information.bristol.ac.uk/files/3020939/osta\_preprint.pdf}{Krassy's paper on psuedo-plateau bursting (huge amounts of good bursting refs in the intro)}}
\label{sec:org2259114}
\subsubsection{Reference}
\label{sec:org445a005}
Osinga, H. M., A. Sherman, and K. Tsaneva-Atanasova. "Cross-currents between biology and mathematics on models of bursting." Bristol Centre for Applied Nonlinear Mathematics preprint 1737 (2011).

\subsubsection{BibTeX}
\label{sec:org2b92c20}
@article\{osinga2011cross,
  title=\{Cross-currents between biology and mathematics on models of bursting\},
  author=\{Osinga, HM and Sherman, A and Tsaneva-Atanasova, K\},
  journal=\{Bristol Centre for Applied Nonlinear Mathematics preprint\},
  volume=\{1737\},
  year=\{2011\}
\}

\subsubsection{Abstract}
\label{sec:org4491f55}
A great deal of work has gone into classifying bursting oscillations,
periodic alternations of spiking and quiescence modeled by fast-slow
systems. In such systems, one or more slow variables carry the fast
variables through a sequence of bifurcations that mediate transitions
between oscillations and steady states. The most rigorous approach is
to characterize the bifurcations found in the neighborhood of a
singularity. Fold/homoclinic bursting, along with most other burst
types of interest, has been shown to occur near a singularity of
codimension three by examining bifurcations of a cubic Lienard system.
Modeling and biological considerations suggest that fold/homoclinic
bursting should be found near fold/subHopf bursting, a more recently
identified burst type whose codimension has not been determined yet.
One would expect that fold/subHopf bursting has the same codimension
as fold/homoclinic bursting, because models of these two burst types
have very similar underlying bifurcation diagrams. However, we are
unable to determine a codimension-three singularity that supports
fold/subHopf bursting. Furthermore, we believe that it is not possible
to find a codimension-three singularity that gives rise to all known
types of bursting. Instead, we identify a three-dimensional slice that
contains all known types of bursting in a partial unfolding of a
doubly-degenerate Bodganov–Takens point, which has codimension four.

\subsubsection{Summary}
\label{sec:orgcfe03f5}
Codim-3 unfoldings aren't enough to explain psuedo-plateau bursting.
To explain it, the paper considers codim-4 unfoldings of a doubly-degenerate Bogdanov Takens bifurcation.
Not only does this add psuedo-plateau bursters to the classification in (3) and (4), but the resulting unfolding also contains all known burster types, making it a very general bursting model.
\begin{enumerate}
\item S1 intro
\label{sec:org8a152dc}
\begin{itemize}
\item Cells can exhibit bursting dynamics
\begin{itemize}
\item These are useful for encouraging calcium buildup, which in turn helps with hormone and neurotransmitter release
\end{itemize}
\item Platea bursting is like VdP oscillator, but with the 'high' state as a limit cycle
\begin{itemize}
\item Cell fires spikes from a depolarised state for a while
\item Good for promoting calcium buildup
\end{itemize}
\item Psuedo-platea bursting is a fairly newly discovered one
\begin{itemize}
\item No LC in the active phase
\item Spikes are actually just the oscillatory transients towards a stable equilibrium (think damped oscillator)
\item This requires a fairly fast slow subsystem, which is a bit weird
\item These burst patterns are yet to be classified; this paper fixes that
\end{itemize}
\end{itemize}

\item S2 bursting normal form
\label{sec:orge323494}
\begin{itemize}
\item Chay-Keizer is a biologically plausible (HH-esque) bursting model
\item Hindmarsh-Rose is a phenomenological bursting model
\item Bertram's unfolding classification built on a deg. BT point unfolding
\begin{itemize}
\item The system equations for its unfolding are presented
\item A slow subsystem model is also given, to facilitate bursting
\end{itemize}
\item A small change (time-reversal) to the d-deg-BT unfolding gives a system that contains all previously categorised bursters, plus more
\begin{itemize}
\item d-deg-BT is our current best guess of a burster normal form
\end{itemize}
\end{itemize}

\item S3 finding a fold/subHopf burster path
\label{sec:org98b8968}
\begin{itemize}
\item d-deg-BT singularity is at the origin
\item b axis outside of 0 contains entirely deg BT points
\begin{itemize}
\item For some reason, the additional d-d-BT bifurcations occur in any fixed neighbourhood around one of these d-BT points, provided b is sufficiently small. WHY?
\item This means we can reduce the parameter space by fixing b small and considering a fixed neighbourhood of 0 in the (mu1,mu2,v) space; the above statement guarantees that we won't lose any interesting bifurcations by doing this
\end{itemize}
\item Take the unit sphere as the fixed neighbourhood. This must be a sufficiently sized neighbourhood to contain the interesting additional bifurcations, as we can find a fold/subHopf path actually on the surface of this sphere
\item Yipee we've found a fold/subHopf burster path in codim4, thus giving us an upper bound on its category codim
\item It can also transition to regular (fold/homoclinic) square wave / plateau bursting, just by shifting the Ca threhsold a bit; this is interesting since the two burst types come from cells that are developmentally and functionally very similar
\item Paths through parameter space are also presented for all the currently known burster types, backing up the claim that this is a good model of neuron bursting
\item Some fold/homoclinic bursters can be perturbed to fold/subHopf bursters with a single parameter change; others can't. This means that there's actually different types of fold/homoclinic burster, even though they're part of the same class. It highlights the difference between classifying bursters from unfoldings, which considers the surrounding bifurcation structure and how that influences cell properties and neighbour cells, and the Rinzel / Izhikevich approach of classifying by the bifurcations that start and stop bursting.
\end{itemize}
\item S4 fold/subHopf codimension
\label{sec:org77baeec}
We now know fold/subHopf bursting can appear in codim4 unfoldings, but can it appear in codim3?
Tl;dr not all codim3 d-BT unfoldings are known, but fold/subHopf doesn't appear in any of the known ones.
\begin{itemize}
\item Fold/subHopf has a region of bistability, with a subHopf on the top branch, ending at a homoclinic
\item Bistable region means the causing singularity must lie on a cusp
\item Hopf and  homoclinic mean we must be at a BT
\item BT + cusp = d-BT point
\end{itemize}

But\ldots{} fold/subHopf can't appear near a d-BT. 

Some long argument that concludes it's probably a codim4 burster.

\item An interesting note from the conclusion
\label{sec:orgb58481c}
Maths has basically only considered planar fast-subsystem bursters.
Cells need to operate in uncertain conditions, and have lots of robustness, so they have a non-minimal set of ion channels, which means they don't actually have planar fast subsystems.
Furthermore, when we couple cells, their fast subsystems grow in dimensionality.
The surface has barely been scratched on these more complex burster types.
Mixed-mode oscillations ideas would be a good place to look into this from.
\end{enumerate}

\subsection{(3) \href{https://link.springer.com/content/pdf/10.1007/BF02460633.pdf}{Classification of bursters according to slow trajectories through the fast-subsystem bifurcation diagram}}
\label{sec:org89b8af4}
\subsubsection{Reference}
\label{sec:org5401e0b}
Bertram, Richard, et al. "Topological and phenomenological classification of bursting oscillations." Bulletin of mathematical biology 57.3 (1995): 413-439.

\subsubsection{BibTeX}
\label{sec:orgce91356}
@article\{bertram1995topological,
  title=\{Topological and phenomenological classification of bursting oscillations\},
  author=\{Bertram, Richard and Butte, Manish J and Kiemel, Tim and Sherman, Arthur\},
  journal=\{Bulletin of mathematical biology\},
  volume=\{57\},
  number=\{3\},
  pages=\{413--439\},
  year=\{1995\},
  publisher=\{Elsevier\}
\}

\subsubsection{Abstract}
\label{sec:org71210df}
We describe a classification scheme for bursting oscillations which
encompasses many of those found in the literature on bursting in
excitable media. This is an extension of the scheme of Rinzel (in
Mathematical Topics in Population Biology, Springer, Berlin, 1987),
put in the context of a sequence of horizontal cuts through a
two-parameter bifurcation diagram. We use this to describe the
phenomenological character of different types of bursting, addressing
the issue of how well the bursting can be characterized giventhe
limited amount of information often available in experimental
settings.

\subsubsection{Summary}
\label{sec:org6307ca5}
Classifies bursters as cuts on a 2-parameter fast-subsystem bifurcation diagram.

\subsection{(4) \href{https://www.asc.ohio-state.edu/golubitsky.4/reprintweb-0.5/output/papers/bursting12.pdf}{Unfolding theory approach to burster classification}}
\label{sec:org2cccd09}
\subsubsection{Reference}
\label{sec:org7b0cd4b}
Golubitsky, Martin, Kresimir Josic, and Tasso J. Kaper. "An unfolding theory approach to bursting in fast-slow systems." Global analysis of dynamical systems (2001): 277-308.

\subsubsection{BibTeX}
\label{sec:orgf21f320}
@article\{golubitsky2001unfolding,
  title=\{An unfolding theory approach to bursting in fast-slow systems\},
  author=\{Golubitsky, Martin and Josic, Kresimir and Kaper, Tasso J\},
  journal=\{Global analysis of dynamical systems\},
  pages=\{277--308\},
  year=\{2001\},
  publisher=\{Inst. Phys.\}
\}

\subsubsection{Abstract}
\label{sec:orgd417264}
Many processes in nature are characterized by periodic bursts of activity separated by intervals of quiescence. In this chapter we describe a method for classifying the types of bursting that occur in models in which variables evolve on two different timescales, ie, fast-slow systems. The classification is based on the observation that the bifurcations of the fast system that lead to bursting can be collapsed to a single local bifurcation, generally of higher codimension. The bursting is recovered as the slow variables periodically trace a closed \ldots{}

\subsubsection{Summary}
\label{sec:org0b2bd82}


(3) explains bursting as cuts through a codim-2 bifurcation diagram. 
This paper takes things a step further, by classifying bursters according to their complexity, in terms of the codimension of the bifurcation in whose unfolding the burster first appears.
It also extends (3)'s classification to include some codim-3 bursters, which covered all known bursters at the time it was written.
(2) takes this even further by studying codim-4 to explain more recently found psuedo-plateau bursters.

\subsection{(5) \href{https://mathematical-neuroscience.springeropen.com/articles/10.1186/s13408-017-0050-8}{A model capable of exhibiting most (hysteresis-loop only) codim-3 bursting behaviours}}
\label{sec:org8927162}
\subsubsection{Reference}
\label{sec:org432943c}
Saggio, Maria Luisa, et al. "Fast–Slow Bursters in the Unfolding of a High Codimension Singularity and the Ultra-slow Transitions of Classes." The Journal of Mathematical Neuroscience 7.1 (2017): 7.

\subsubsection{BibTeX}
\label{sec:org5665555}
@article\{saggio2017fast,
  title=\{Fast--Slow Bursters in the Unfolding of a High Codimension Singularity and the Ultra-slow Transitions of Classes\},
  author=\{Saggio, Maria Luisa and Spiegler, Andreas and Bernard, Christophe and Jirsa, Viktor K\},
  journal=\{The Journal of Mathematical Neuroscience\},
  volume=\{7\},
  number=\{1\},
  pages=\{7\},
  year=\{2017\},
  publisher=\{SpringerOpen\}
\}

\subsubsection{Abstract}
\label{sec:orgbdce4ec}
Bursting is a phenomenon found in a variety of physical and biological
systems. For example, in neuroscience, bursting is believed to play a
key role in the way information is transferred in the nervous system.
In this work, we propose a model that, appropriately tuned, can
display several types of bursting behaviors. The model contains two
subsystems acting at different time scales. For the fast subsystem we
use the planar unfolding of a high codimension singularity. In its
bifurcation diagram, we locate paths that underlie the right sequence
of bifurcations necessary for bursting. The slow subsystem steers the
fast one back and forth along these paths leading to bursting
behavior. The model is able to produce almost all the classes of
bursting predicted for systems with a planar fast subsystem.
Transitions between classes can be obtained through an ultra-slow
modulation of the model’s parameters. A detailed exploration of the
parameter space allows predicting possible transitions. This provides
a single framework to understand the coexistence of diverse bursting
patterns in physical and biological systems or in models.

\subsubsection{Summary}
\label{sec:org891446c}
Extends Krassy's work (sort of?) by providing a slow subsystem to complete a model of hysteresis loop codim-3 bursters.
Model will be useful for CBC.

\subsection{\href{https://scholar.google.com/scholar?hl=en\&as\_sdt=0\%2C5\&q=global+study+of+a+family+of+cubic+lienard+equations\&btnG=}{Global study of a family of cubic lienard equations}}
\label{sec:orgd3ebf51}
\subsubsection{Reference}
\label{sec:orgc7dedc9}
Khibnik, Alexander I., Bernd Krauskopf, and Christiane Rousseau. "Global study of a family of cubic Liénard equations." Nonlinearity 11.6 (1998): 1505.

\subsubsection{BibTeX}
\label{sec:orga32a4ca}
@article\{khibnik1998global,
  title=\{Global study of a family of cubic Li\{$\backslash$'e\}nard equations\},
  author=\{Khibnik, Alexander I and Krauskopf, Bernd and Rousseau, Christiane\},
  journal=\{Nonlinearity\},
  volume=\{11\},
  number=\{6\},
  pages=\{1505\},
  year=\{1998\},
  publisher=\{IOP Publishing\}
\}

\subsubsection{Abstract}
\label{sec:orgdf8362f}
We derive the global bifurcation diagram of a three-parameter family
of cubic Liénard systems. This family seems to have a universal
character in that its bifurcation diagram (or parts of it) appears in
many models from applications for which a combination of hysteretic
and self-oscillatory behaviour is essential. The family emerges as a
partial unfolding of a doubly degenerate Bogdanov-Takens point, that
is, of the codimension-four singularity with nilpotent linear part and
no quadratic terms in the normal form. We give a new presentation of a
local four-parameter bifurcation diagram which is a candidate for the
universal unfolding of this singularity.

\subsubsection{Summary}
\label{sec:org7758f7a}
Krassy's model uses a cubic Lienard equation as the fast subsystem. 
This paper derives the global bifurcation diagram of the system.
It's hard.
The paper contains some nice analytical descriptions of where in parameter space some bifurcations occur, but it also contains some particularly confusing theorems and proofs and stuff.

\subsection{\href{https://scholar.google.com/scholar?hl=en\&as\_sdt=0\%2C5\&q=fast+subsystem+bifurcations+in+a+slowly+varying+lienard+sysetem+exhibiting+bursting\&btnG=}{Fast subsystem bifurcations in a slowly varying lienard system exhibiting bursting}}
\label{sec:org3e08832}
\subsubsection{Reference}
\label{sec:org4f95d66}
Pernarowski, Mark. "Fast subsystem bifurcations in a slowly varying Lienard system exhibiting bursting." SIAM Journal on Applied Mathematics 54.3 (1994): 814-832.

\subsubsection{BibTeX}
\label{sec:orgc40250c}
@article\{pernarowski1994fast,
  title=\{Fast subsystem bifurcations in a slowly varying Lienard system exhibiting bursting\},
  author=\{Pernarowski, Mark\},
  journal=\{SIAM Journal on Applied Mathematics\},
  volume=\{54\},
  number=\{3\},
  pages=\{814--832\},
  year=\{1994\},
  publisher=\{SIAM\}
\}

\subsubsection{Abstract}
\label{sec:orga15b1a2}
A perturbed Liénard differential system is examined using local
stability and Hopf bifurcation analyses, asymptotic techniques, and
Melnikov's method. The results of these analyses are applied to a
simple cubic model that exhibits a variety of different oscillatory
behaviors for different parameter values. For a bounded region in
(fast) parameter space, the model exhibits square-wave bursting
patterns analogous to the bursting electrical activity observed in
pancreatic ,\$\(\beta\) \$-cells. Under certain hypotheses, solutions of the
cubic model are known to have square-wave patterns. By using the
theory developed for the more general Liénard system, each hypothesis
is shown to correspond to a curve in parameter space. Together, the
curves bound a region in which the model exhibits square-wave bursting
patterns. Since the model is simple, the curves that bound this region
can all be determined analytically.

\subsubsection{Summary}
\label{sec:orged328a9}
Contains some nice references about bursting sytems, and has a nice model derivation.
Also contains some rather impenetrable mathematical analysis.
\end{document}
