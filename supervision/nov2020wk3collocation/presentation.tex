% Created 2020-11-30 Mon 09:33
% Intended LaTeX compiler: pdflatex
\documentclass[presentation]{beamer}
\usepackage[utf8]{inputenc}
\usepackage[T1]{fontenc}
\usepackage{graphicx}
\usepackage{grffile}
\usepackage{longtable}
\usepackage{wrapfig}
\usepackage{rotating}
\usepackage[normalem]{ulem}
\usepackage{amsmath}
\usepackage{textcomp}
\usepackage{amssymb}
\usepackage{capt-of}
\usepackage{hyperref}
\usetheme{UoB}
\author{Mark Blyth}
\date{\textit{[2020-11-30 Mon]}}
\title{Collocation for control-based continuation}
\hypersetup{
 pdfauthor={Mark Blyth},
 pdftitle={Collocation for control-based continuation},
 pdfkeywords={},
 pdfsubject={},
 pdfcreator={Emacs 27.1 (Org mode 9.3)}, 
 pdflang={English}}
\begin{document}

\maketitle

\section{Intro}
\label{sec:org96b2791}
\begin{frame}[label={sec:orgf6fc485}]{Summary}
\begin{itemize}
\item Previous results: discussing how we could use BSplines for discretisation
\begin{itemize}
\item Lower-dimensional discretisations mean more speed
\end{itemize}
\end{itemize}
\vfill
\begin{itemize}
\item Results: it's incredibly numerically finicky
\begin{itemize}
\item Numerical solvers struggle to get accurate results
\item Takes a lot of trial and error to get close to a correct solution curve
\end{itemize}
\end{itemize}
\vfill
\begin{itemize}
\item This time: can we use an alternative discretisation?
\end{itemize}
\end{frame}

\section{CBC problem formulation}
\label{sec:orgd4e067e}
\begin{frame}[label={sec:org33993e1}]{CBC goal}
Say we have a system \(\dot{y} = g(y,\mu)\)
\vfill
\begin{itemize}
\item Let \(y_s(\cdot/T_s)\) be a \(T_s\)-periodic solution, for parameter \(\mu_s\)
\end{itemize}
\vfill
\begin{itemize}
\item Goal: given \((y_0, \mu_0, T_0)\), trace out solution family \(\Gamma = \{(y_s, \mu_s, T_s):s\in\mathbb{R}\}\)
\end{itemize}
\vfill
\begin{itemize}
\item We seek noninvasive control targets to achieve this
\begin{itemize}
\item Noninvasive = target that can be tracked with zero control action
\item Zero control action = system operating under its free dynamics
\end{itemize}
\end{itemize}
\end{frame}

\begin{frame}[label={sec:orgec896d5}]{Finding noninvasive control}
\begin{itemize}
\item The controlled system maps control target \(y_\text{in}\) to output \(y_\text{out}\)
\end{itemize}
\vfill
\begin{itemize}
\item Denote this input-output, or IO map, by \(Y(y_s, T_s, \mu_s)\)
\end{itemize}
\vfill
\begin{itemize}
\item Assume a proportional controller
\end{itemize}
\vfill
\begin{itemize}
\item A fixed-point of \(Y\) is noninvasive
\begin{itemize}
\item Fixed point means system output exactly matches control target
\item Zero tracking error means zero proportional feedback
\item Zero proportional feedback means controller is switched off
\end{itemize}
\end{itemize}
\vfill
We can use the IO map fixed-point-problem for our continuation equations!
\end{frame}

\begin{frame}[label={sec:orgd76f1a1}]{Solving the IO map}
\begin{itemize}
\item We want to solve the fixed-point problem \(y^*_s = Y(y^*_s, T_s, \mu_s)\)
\begin{itemize}
\item IO map is evaluated by running the controlled system; slow!
\end{itemize}
\end{itemize}
\vfill
\begin{itemize}
\item Continuous problem; not numerically tractable
\begin{itemize}
\item To apply standard numerical methods, we must first discretise the system
\end{itemize}
\end{itemize}
\vfill
\begin{itemize}
\item We seek some discretisation of the map
\begin{itemize}
\item Goal: find a finite-dimensional problem that we can pass to a numerical solver
\item Select a discretised problem that also solves the continuous problem
\end{itemize}
\end{itemize}
\end{frame}

\section{Collocation for CBC}
\label{sec:org9cf54cc}
\begin{frame}[label={sec:org6153527}]{Discretisation with Galerkin projections}
Discretisation method used by all current CBC applications
\vfill
\begin{itemize}
\item Take some signal \(y(t)\in\mathbb{R}\)
\end{itemize}
\vfill
\begin{itemize}
\item Let \(\beta_i\), \(B_i(t)\), \(i=-n,\dots,0,\dots,n\) be the coefficients, basis functions of its \(n-\)truncated Fourier series
\end{itemize}
\vfill
\begin{itemize}
\item \(\beta_i\) is our discretisation, and \(\tilde{y} = \sum_{-n}^n \beta_iB_i(t)\) is our reconstruction of \(y\)
\end{itemize}
\vfill
\begin{itemize}
\item If \(\tilde{y} = Y(\tilde{y}, T_s, \mu_s)\), then \(\beta_i^\text{in} = \beta_i^\text{out}\)
\end{itemize}
\vfill
\begin{itemize}
\item To solve the fixed-point problem, we find the basis function coefficients that remain unchanged when passed through the IO map
\begin{itemize}
\item \(2n+1\)-dimensional problem; numerically tractable!
\end{itemize}
\end{itemize}
\end{frame}

\begin{frame}[label={sec:org6a1ff6c}]{Issues with current CBC discretisation}
\begin{itemize}
\item Evaluation of continuation equations is \emph{slow}
\begin{itemize}
\item Newton iterations require a Jacobian, which requires finite differences
\item This means we need to run physical system to convergence, many times
\end{itemize}
\end{itemize}
\vfill
\begin{itemize}
\item We can only find the noninvasive solution \(y\) using Galerkin discretisation when \(y\in span\{B_1, B_2, \dots B_m\}\)
\begin{itemize}
\item This limits our choice of basis functions
\item We might still be able to find an approximate solution when this doesn't hold, but I wouldn't know how to prove or disprove this
\end{itemize}
\end{itemize}
\end{frame}

\begin{frame}[label={sec:orgd6e38b4}]{BSpline discretisation}
\begin{itemize}
\item We can speed up prediction-correction by reducing number of evaluations
\begin{itemize}
\item Easily achievable with lower-dimensional discretisation
\end{itemize}
\end{itemize}
\vfill
\begin{itemize}
\item One option: use more `efficient' basis functions
\begin{itemize}
\item A Fourier basis is inefficient for neuronal signals; can we find more efficient basis functions?
\end{itemize}
\end{itemize}
\vfill
\begin{itemize}
\item Discretisation with BSplines is very numerically difficult; hard to find an accurate solution, even when playing with
\begin{itemize}
\item Continuation stepsize
\item Finite differences stepsize
\item Number of basis functions
\item Convergence tolerance
\end{itemize}
\end{itemize}
\vfill
\begin{itemize}
\item Another option: can we use non-Fourier basis functions with another discretisation method?
\end{itemize}
\end{frame}

\begin{frame}[label={sec:org0919a72}]{Collocation}
Instead of solving the problem exactly, by requiring the input and output discretisations to be exactly equal, we could solve it approximately
\vfill
\begin{itemize}
\item Collocation defines a discrete approximation of the problem, that we can solve exactly
\end{itemize}
\vfill
\begin{itemize}
\item We can always find an approximate solution when using collocation \emph{[I think]}
\begin{itemize}
\item Collocation solution will \emph{[hopefully]} be easier to find
\item Conjecture: the collocation solution will be identical to the Galerkin projection solution in cases where Galerkin projection works
\end{itemize}
\end{itemize}
\vfill
\begin{itemize}
\item Collocation discretisation -- hopefully less numerically fiddly
\item Non-fourier basis functions -- lower-dimensional discretisation: faster!
\end{itemize}
\end{frame}

\begin{frame}[label={sec:org50baf68}]{Collocation setup}
\begin{itemize}
\item We approximate the solution with \(y_\text{in} = \sum \beta_i B_i(t)\), for some basis functions \(B_i(t)\)
\end{itemize}
\vfill
\begin{itemize}
\item We split the signal period into a mesh \(\left[\xi_1=0 < \xi_2 < \dots < \xi_{n}=T_s\right]\)
\end{itemize}
\vfill
\begin{itemize}
\item We solve for \(\beta_i\) such that \(y_\text{in}(\xi_i) = y_\text{out}(\xi_i)\)
\begin{itemize}
\item We also add any phase constraints, periodicity constraints into the system
\item Here, \(y_\text{out}(t) = Y(\sum\beta_iB_i(t), T_s, \mu_s)\)
\end{itemize}
\end{itemize}
\vfill
\begin{itemize}
\item Collocation solution solves the fixed-point problem exactly, at the collocation meshpoints
\begin{itemize}
\item We assume it's a good approximation between meshpoints
\item Resulting \(\beta_i\) give our signal discretisation for continuation
\item Resulting function \(\sum\beta_i B_i(t)\) gives a control target
\end{itemize}
\end{itemize}
\end{frame}

\begin{frame}[label={sec:orgf31c4a6}]{Comparison of methods}
\begin{itemize}
\item Galerkin methods require us to translate from signal to discretisation, eg. using FFT
\item Collocation does not, which offers a slight speed-up
\end{itemize}
\vfill
\begin{itemize}
\item Galerkin basis functions will help filter noise off
\item Collocation offers no noise-filtering
\end{itemize}
\vfill
\begin{itemize}
\item Collocation automatically aims to find the best approximation; it should hopefully be robust to cases where a solution can only be found to a limited accuracy
\item Galerkin methods aim for a correct solution straight away; this makes them harder to apply when we're limited in solution accuracy, eg. by not having enough Fourier harmonics
\end{itemize}
\end{frame}

\begin{frame}[label={sec:orgc1677c0}]{Potential collocation pitfalls}
\begin{itemize}
\item Collocation is not noise-robust
\begin{itemize}
\item We're searching for equality between input and output signals
\item If we have measurement error, output values at the meshpoints are a random variable
\item Instead of searching for equality, we would need a maximum-likelihood estimation on \(\beta_i\)
\item Alternatively, use a surrogate model to filter the noise off!
\end{itemize}
\end{itemize}
\vfill
\begin{itemize}
\item Collocation finds an approximate solution
\begin{itemize}
\item We can only guarantee the discretised problem to be solved at the meshpoints
\item Collocation solution may deviate from true solution between meshpoints, in which case we wouldn't have noninvasive control
\item Could implement a solution-checker, by measuring the distance between input and output functions
\end{itemize}
\end{itemize}
\end{frame}

\section{Where next?}
\label{sec:org60f7ef2}
\begin{frame}[label={sec:orgd77df3c}]{Next steps}
\begin{itemize}
\item Code up a collocation CBC simulation
\end{itemize}
\vfill
\begin{itemize}
\item See if it works!
\end{itemize}
\vfill
\begin{itemize}
\item Perform numerical experiments to compare its solution accuracy against Galerkin discretisation
\end{itemize}
\vfill
\begin{itemize}
\item Test its noise-robustness with surrogates
\end{itemize}
\end{frame}
\end{document}
