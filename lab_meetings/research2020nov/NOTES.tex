% Created 2020-11-04 Wed 13:25
% Intended LaTeX compiler: pdflatex
\documentclass[11pt]{article}
\usepackage[usenames,dvipsnames,svgnames,table]{xcolor}
\newenvironment{note}{\color{red}\bfseries ZZZ}


\usepackage[utf8]{inputenc}
\usepackage[T1]{fontenc}
\usepackage{graphicx}
\usepackage{grffile}
\usepackage{longtable}
\usepackage{wrapfig}
\usepackage{rotating}
\usepackage[normalem]{ulem}
\usepackage{amsmath}
\usepackage{textcomp}
\usepackage{amssymb}
\usepackage{capt-of}
\usepackage{hyperref}
\author{Mark Blyth}
\date{\textit{[2020-11-04 Wed]}}
\title{MARK'S NOTES THAT MUSTN'T BE PRESENTED}
\hypersetup{
 pdfauthor={Mark Blyth},
 pdftitle={MARK'S NOTES THAT MUSTN'T BE PRESENTED},
 pdfkeywords={},
 pdfsubject={},
 pdfcreator={Emacs 27.1 (Org mode 9.3)}, 
 pdflang={English}}
\begin{document}

\maketitle

\section{Background}
\label{sec:orgac975df}
\subsection{Today's agenda}
\label{sec:org0a28696}

Been writing my annual review recently.
This presentation is aimed at giving an overview of that, as it nicely sums up what I've been up to so far, and what I want to get up to in the future.
Wasn't quite sure how much depth to go into, so if you have any questions, stop me at any point and ask!

\begin{itemize}
\item \textbf{A brief summary of things}

\item More CBC

\item Results so far

\item Current work
\end{itemize}


\section{A brief summary of things}
\label{sec:org60c2056}
\subsection{What am I doing?}
\label{sec:org7cd7578}
\begin{itemize}
\item Neurons are interesting
\begin{itemize}
\item Biological perspective: excitable cell; lots of reaction networks that produce surprisingly subtle functionalities within the cells
\item Philosophical perspective: individual cells are phenomenologically simple, yet collect them together into a brain and suddenly consiousness emerges. How? Why? Where from?
\item Mathematical perspective: neurons recieve their computational abilities from their intricate bifurcation structure, making them an excellent problem to study in nonlinear dynamics
\end{itemize}

\item Nonlinear dynamics teaches us lots about neurons
\begin{itemize}
\item We have lots of good neuron models
\item These models predict all sorts of different behaviours, and can, and do!, explain experimental observations
\item As mathematicians, we tend to focus entirely on the model; even if it's not a good electrophysiological representation of a neuron, it doesn't necessarily matter as long as it has some interesting dynamics
\end{itemize}

\item Models are wrong
\begin{itemize}
\item All models are wrong, but some are useful
\item The best models of the neurons are the cells themselves
\item What if we could perform our favourite analyses not on phenomenological models, but on the cells themselves?
\item Double win: we get interesting nonlinear dynamics to play with, \emph{and} the results are biologically useful!
\end{itemize}
\end{itemize}

\subsection{How am I doing it?}
\label{sec:org1be5909}
\begin{itemize}
\item Models are often analysed using numerical continuation
\begin{itemize}
\item Nonlinear dynamics are usually hard to study
\item Instead of pen-and-papering everything, we can outsource the labour to a computer
\item Numerical continuation lets us take something we can easily find, eg. equilibria, and track them to spot things that are harder to find, eg. bifurcations
\item Very powerful tool; very widely used
\end{itemize}

\item Numerical continuation needs a model
\begin{itemize}
\item The model encodes our assumptions about how the cells work
\item If the assumptions are bad, the model will be unrepresentative, and the continuation results won't match reality
\item What if instead, we could run the same analysis without a model?
\end{itemize}

\item Control-based continuation doesn't
\begin{itemize}
\item CBC let's us perform continuation experiments without the need for a model
\item If a system is observable and controllable, it's CBCable
\item My overall goal is to study neurons using CBC
\end{itemize}
\end{itemize}

\subsection{What needs to be done?}
\label{sec:org4db8c52}
\begin{itemize}
\item Make it fast
\begin{itemize}
\item Neurons have a finite lifespan
\item We could leave a mechanical system oscillating for days on end, but if we tried that with neurons, they would die
\item There's some very slow steps in CBC; new mathematical techniques might allow some big speed-ups here
\end{itemize}

\item Make it noise-robust
\begin{itemize}
\item CBC relies on experimental meausurements, which are often subject to lots of measurement noise
\item The CBC algo, solvers, etc. must therefore be robust in the face of noise
\item Furthermore, we could reasonably decide to model the neuron cells themselves as being a random dynamical system
\item Considerations of stochastic dynamics would be a whole new kettle of fish, and potentially a very interesting and useful one!
\item Stochastic continuation would be very useful eg. in aeroelastic flutter; perhaps treat turbulent fluid flow as a random influence, and see how the deterministic results change when randomness is considered
\end{itemize}

\item Make it happen
\begin{itemize}
\item Everything so far has been in silico
\item That means I've been simulating mathematical models, instead of using real cells
\item Of course, the end goal is to move beyond models, so I'll need to consider how these methods would need to change for an experiment
\end{itemize}
\end{itemize}

\subsection{How are those TODOs progressing?}
\label{sec:org03f83a2}
\begin{itemize}
\item Efficiency
\begin{itemize}
\item Current work; lots of problems, lots of progress
\begin{itemize}
\item Currently studying how alternative discretisations could be employed to speed things up
\item Lots going wrong, but in research that seems to be a sign of good progress\ldots{}
\item Very recently demonstrated it working for the first time; still lots of open questions, lots of issues about it not working how it should
\end{itemize}
\end{itemize}

\item Noise-robustness
\begin{itemize}
\item One paper under review
\begin{itemize}
\item Has been discussed in my previous research update
\item Essential idea is to take time-series data, fit a nonparametric regression model to it, and use that model in place of the data
\item If we choose the models well, they'll accurately separate signal from noise
\item Acts like an adaptive filter, allowing us to keep all the signal information, and none of the random fluctuations
\item Downside is that I'd consider it to have a very limited set of usage cases; most of the time it wouldn't be a useful thing to do
\end{itemize}
\item Other ideas under consideration
\begin{itemize}
\item Idea that's been mentioned a while ago but I'm yet to do anything with
\item Replace continuation equations with a different set of equations that encodes exactly the same thing
\item New equations should be noise-robust, and robust against discretisation approximation errors
\item Big bonus: new equations can be solved efficiently with gradient-free methods, which will improve point 1 of efficiency
\item Second big bonus: efficient solver methods are probabilistic, so will likely perform better in the face of noise
\end{itemize}
\end{itemize}

\item Experiments
\begin{itemize}
\item Minireview of literature
\begin{itemize}
\item Had a look at microfluidic methods, electrode-based methods
\item Compared single-cell and multicell microfludidic approaches, in terms of captured nonlinear dynamics, and experimental viability
\end{itemize}
\item Some practical experience
\begin{itemize}
\item Helped build some other microfluidics in the clean room, to get an understanding of how it's done and what they look like
\end{itemize}
\end{itemize}
\end{itemize}

\section{More CBC}
\label{sec:org007c0dd}
\subsection{Today's agenda}
\label{sec:orgc0ebdd9}
\begin{itemize}
\item A brief summary of things

\item \textbf{More CBC}

\item Results so far

\item Current work
\end{itemize}



\subsection{Control-based continuation}
\label{sec:orgbce1316}
This has been explained many times before, in the lab group meetings, so I'll keep things brief here and give only the very high-level overview.

\begin{itemize}
\item CBC works by tracking non-invasive control targets
\begin{itemize}
\item This is a control target that corresponds to something the system was already doing
\item The controller therefore doesn't change how the system was behaving
\item The only difference between the controlled and uncontrolled system, in this case, is that unstable equilibria and periodic orbits are stabilised, so they become visible
\item Since they become visible, we can see how they move when we change a parameter
\item This `seeing how they move' step is done with a continuation algo, in much the same way as with `normal' continuation
\end{itemize}

\item It has been tested on `nice' systems, but biological systems aren't nice
\begin{itemize}
\item By `nice', I mean\ldots{}
\begin{itemize}
\item Deterministic: no randomness within the dynamics
\item Low noise in the observations
\item `Simple' signals, with few high-frequency components; well-approximated by a shifted sine wave, as the dynamics are only weakly nonlinear
\end{itemize}
\item Biological systems, such as neurons, don't follow the niceness rules
\begin{itemize}
\item Dynamics may be stochastic
\item There's often quite a lot of measurement noise
\item The signals are from a strongly nonlinear oscillator; have lots of high-frequency components; very very different to a sine wave
\end{itemize}
\end{itemize}

\item Discretisation is a key part of this
\begin{itemize}
\item Finding and tracking the noninvasive control targets requires us to solve for the fixed point of some map
\item This map maps from a function to a function
\item Instead of working with the continuous, infinite-dimensional map, we instead approximate it with a finite, lower-dimensional map, in a process called discretisation
\item We then attempt to solve the finite-dimensional equations using standard numerical methods, and hope that the results of the discretised case correspond to a solution of the original, continuous problem
\item I don't know if these hopes are actually valid, and nor would I know how to prove that
\end{itemize}
\end{itemize}

I'll get back to the topic of discretisation later, as that's a key part of my current research.

\section{Results so far}
\label{sec:org2e0036c}
\subsection{Today's agenda}
\label{sec:org75eaf77}

I have a couple of results so far.
These are a tutorial paper, currently under review, and a conference paper, also currently under review.
I'll talk briefly about these here.

\begin{itemize}
\item A brief summary of things

\item More CBC

\item \textbf{Results so far}

\item Current work
\end{itemize}



\subsection{Paper 1: `Tutorial of \dots{}'}
\label{sec:org3a79022}
\begin{center}
Tutorial of numerical continuation for systems and synthetic biology
\end{center}

\begin{itemize}
\item Already mentioned that numerical continuation is a very standard, widespread tool in nonlinear dynamics
\item I spent a long time playing around with different continuation softwares while learning about neuronal dynamics
\item This paper aims to bridge the gap between biologists and mathematicians
\item Aims to expose numerical continuation, and some key ideas from nonlinear dynamics, to researchers without a nonlinear dynamics background
\item Uses lots of examples to give a conceptual, high-level overview of the topic, so that readers can go on to understand work that builds on continuation, bifurcation theory, and so on
\end{itemize}

\subsection{Paper 2: `Bayesian local \dots{}'}
\label{sec:org29e5bc9}

\begin{center}
Bayesian local surrogate models for the control-based continuation of multiple-timescale systems
\end{center}

\begin{itemize}
\item Noise-robustness is important in CBC
\begin{itemize}
\item A point I touched on earlier
\item Lots of ways noise could enter the system
\item We can't treat noise as a negligable side-issue; we must pay active attention to dealing with it
\item This paper aims to do that, by using an adaptive filtering method
\end{itemize}

\item Surrogate modelling is a possible route towards noise-robust experiments
\begin{itemize}
\item Instead of running CBC using raw noise-corrupted experimental measurements, we could instead filter the data
\item Filtering needs to be done carefully!
\begin{itemize}
\item If we whack the data through a simple low-pass filter, we'll cut off the all-important high-frequency information
\end{itemize}
\item Instead of using a simple low-pass filter, let's use a complicated one!
\item Take the time series data
\item Fit a nonparametric regression model to it
\item With a well-chosen model, we can filter out the noise
\item This works because a statistical model of the time series will describe the data as signal + noise; we fit a model to the signal part of it, and throw away the noise residuals
\item We can then use the surrogate in place of the original data, to perform whichever analyses we wanted to do
\item We can discretise the surrogate more accurately than we can the original data
\end{itemize}
\end{itemize}

\section{Current work}
\label{sec:orgd2f495b}
\subsection{Today's agenda}
\label{sec:orgfac0ff3}

That's a very brief summary of what I'm trying to acheive and what I've finished working on so far.
Now we're going to move on to what I'm currently working on, and what I want to achieve next.

\begin{itemize}
\item A brief summary of things

\item More CBC

\item Results so far

\item \textbf{Current work}
\end{itemize}



\subsection{Periodic splines discretisation}
\label{sec:org8f57c54}
\begin{itemize}
\item Discretisation is important
\begin{itemize}
\item The continuation equations are infinite-dimensional
\item We have a map that takes a function as its input, and gives a function as its output
\item We're searching for a function that remains unchanged when passed through this map
\item Such a function exists, but to find it we need to reduce the problem to something more tractable
\item This is where discretisation comes in
\item We approximate the infinite-dimensional problem with a finite-dimensional problem
\item The finite-dimensional problem is then tackled using standard numerical methods
\end{itemize}

\item Efficiency is also important
\begin{itemize}
\item Another point I touched on earlier
\item We want the experiments to run quickly, so that our cells survive
\item A big issue in making this happen is the gradient step
\item We're using numerical solvers on our continuation equatiosn
\item Virtually all numerical solvers require a gradient to work
\item Finding the gradient of an experimental system requires finite differences
\item This means perturbing the input vector slightly, and noting how the output vector changes
\item More elements in the input vector means more time spent perturbing, running to convergence, and measuring
\item This all takes time
\item Therefore, to speed things up, we want the fewest elements possible in our vector
\item IE. we need a low-dimensional discretisation
\end{itemize}

\item Splines could be efficient discretisors
\begin{itemize}
\item Neuronal signals have lots of high-frequency energy
\item This HF energy is what gives the signals their spiking shapes
\item The issue with this is that Fourier discretisation, as used so far in CBC, would require huge numbers of Fourier harmonics
\item This gives a big discretisation, which means slow finite differences, and inefficient experiments
\item Spline models are smooth piecewise-polynomial models
\item We can make them periodic, too
\item We can fit very complex curves by connecting the dots with pieces of polynomial
\item This makes them a good discretisor candidate!
\item Ideal: replace the Fourier basis functions with spline basis functions, for a novel discretisation
\item This is my current work
\item Bonus: spline discretisation might be more noise-robust, too
\item Or it might not be
\end{itemize}
\end{itemize}

\subsection{Current issues}
\label{sec:org98983d2}

\begin{itemize}
\item Newton solvers don't converge on a solution
\begin{itemize}
\item The discretised continuation equations are a set of nonlinear equations whereby we put in some vector, and get a vector as an output
\item If the vector represents a noninvasive control target, the continuation equations will give the zero-vector as their output
\item Therefore, the Newton iterations seek some input vector that solves these equations, and we know we've solved them when we plug the solution in and get zero out
\item Unfortunately, the Newton iterations converge on a vector that doesn't actually solve the system
\item The steps become negligable, but the resulting vectors don't solve the equations
\item This is a problem, as it means the accepted results are wrong
\item Nevertheless, a solution must exist, as it can be found using non-Newton solving methods
\item My current hunch is that the issues are arising as a result of difficulties in calculating the gradient accurately
\end{itemize}

\item The solution curve becomes numerically unstable
\begin{itemize}
\item As seen in the diagram, the solution jumps shortly after the second fold
\item The real (analytic) solution does not jump, so clearly this is the result of something going wrong with the numerics
\item My hope is that if I can find a solution to the gradient problem, it will fix this too
\end{itemize}

\item Current work is therefore trying to fix this, so that I can test out the novel discretisation methods on neuronal data; all my discretisor experiments so far are on the weakly nonlinear Duffing oscillator
\end{itemize}
\end{document}
