% Created 2020-09-07 Mon 15:36
% Intended LaTeX compiler: pdflatex
\documentclass[presentation]{beamer}
\usepackage[utf8]{inputenc}
\usepackage[T1]{fontenc}
\usepackage{graphicx}
\usepackage{grffile}
\usepackage{longtable}
\usepackage{wrapfig}
\usepackage{rotating}
\usepackage[normalem]{ulem}
\usepackage{amsmath}
\usepackage{textcomp}
\usepackage{amssymb}
\usepackage{capt-of}
\usepackage{hyperref}
\usetheme{UoB}
\date{\textit{[2020-09-07 Mon]}}
\title{Proposal on machine learning via DS}
\hypersetup{
 pdfauthor={},
 pdftitle={Proposal on machine learning via DS},
 pdfkeywords={},
 pdfsubject={},
 pdfcreator={Emacs 27.1 (Org mode 9.3)}, 
 pdflang={English}}
\begin{document}

\maketitle

\section{Intro}
\label{sec:org85e9ae1}
\begin{frame}[label={sec:org46b4937}]{What and why?}
\begin{itemize}
\item ML is `fancy' model fitting
\end{itemize}

\vfill
\begin{itemize}
\item ODE solutions could be a rich source of models to fit
\end{itemize}
\end{frame}

\section{Formulation}
\label{sec:org0773ed5}
\begin{frame}[<+->][label={sec:org37c23ae}]{Section 1: formulation}
\begin{itemize}
\item Let \(z(T,x)\) be a solution to an ODE, evaluated at time \(T\), initial condition \(x\)
\end{itemize}
\vfill
\begin{itemize}
\item \(\frac{\mathrm{d}z}{\mathrm{d}t} = f(A(t), z),~~z(0)=x\)
\end{itemize}
\vfill
\begin{itemize}
\item \(u(x) = \mathbf{a}\cdot \mathbf{z} + b\)
\end{itemize}
\vfill
\begin{itemize}
\item \(\mathrm{argmin}_{\mathbf{a},b,A} \sum(y_i - u(x_i))^2\)
\end{itemize}
\end{frame}

\begin{frame}[label={sec:org43bb638}]{Finding a controller}
\begin{itemize}
\item \(\mathrm{performance} = \int \mathrm{error}^2 \mathrm{d}\mu(x)\)
\end{itemize}

\vfill
\begin{itemize}
\item \(\frac{\mathrm{d~performance}}{\mathrm{d}A} = \int \frac{\mathrm{d~error}^2}{\mathrm{d}A} \mathrm{d}\mu(x)\)
\end{itemize}

\vfill
\begin{itemize}
\item \(\frac{\mathrm{d~perturbation}}{\mathrm{d}\tau} = J_z \mathrm{perturbation}\)
\end{itemize}

\vfill
\begin{itemize}
\item \(\frac{\mathrm{d~output}}{\mathrm{d}A} = \mathrm{perturbation}(T)\)
\end{itemize}
\end{frame}

\section{DNN Connection}
\label{sec:org7298f63}
\begin{frame}[label={sec:orgb86db5e}]{Connection to deep NNs}
\begin{itemize}
\item Deep NNs are a dynamical system that can change dimension
\end{itemize}
\vfill
\begin{itemize}
\item Continuous NNs cannot change dimensions
\end{itemize}
\vfill
\begin{itemize}
\item Continuous NNs can overcome issues with training deep NNs
\end{itemize}
\end{frame}

\begin{frame}[label={sec:orgb21df23}]{Connection to deep resnets}
\begin{itemize}
\item Residual neural networks overcome vanishing gradients by selectively omitting layers
\end{itemize}
\vfill
\begin{itemize}
\item The dynamical systems viewpoint explains why this should help training
\end{itemize}
\vfill
\begin{itemize}
\item Resnets learn an Euler-discretisation of an ODE
\end{itemize}
\end{frame}
\section{Representability and controllability}
\label{sec:orgc89a5d0}
\begin{frame}[label={sec:org9b3d5bf}]{Representability and controllability}
\begin{itemize}
\item We need to be sure that the flow map can process data as desired
\end{itemize}
\vfill
\begin{itemize}
\item This is a problem of controllability
\end{itemize}
\vfill
\begin{itemize}
\item Idealised problem: can the flow-map model arbitrary mappings on the data?
\end{itemize}
\end{frame}

\section{Extensions}
\label{sec:orgac5e824}
\begin{frame}[label={sec:orgf79dd24}]{Continuum in space}
\begin{itemize}
\item PDE models are useful when we have spatially structured data
\end{itemize}
\vfill
\begin{itemize}
\item Using a convolutional kernel gives CNN-like behaviours
\end{itemize}
\end{frame}


\begin{frame}[label={sec:orgce2b76e}]{Constraints, structure, and regularisation}
\begin{itemize}
\item We could add constraints to the system
\end{itemize}
\vfill
\begin{itemize}
\item We could add structure to the ODEs
\end{itemize}
\vfill
\begin{itemize}
\item We could add regularisation terms
\end{itemize}
\end{frame}


\begin{frame}[label={sec:orgdcfaac9}]{Clustering and density estimation}
\begin{itemize}
\item A clustering model is presented
\begin{itemize}
\item Not quite sure how it relates to the rest of the paper?
\end{itemize}
\end{itemize}
\vfill
\begin{itemize}
\item Density estimation can also be performed with the flow-map framework
\end{itemize}
\end{frame}

\begin{frame}[label={sec:orgbb47720}]{Paper suggestion}
Raissi, Maziar, Paris Perdikaris, and George E. Karniadakis. "Physics-informed neural networks: A deep learning framework for solving forward and inverse problems involving nonlinear partial differential equations." Journal of Computational Physics 378 (2019): 686-707.
\vfill

Any volunteers?
\end{frame}
\end{document}
